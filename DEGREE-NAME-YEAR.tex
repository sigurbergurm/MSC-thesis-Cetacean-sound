%% ---------------------------------------------------------------
%% $URL: https://repository.cs.ru.is/svn/thesis-template/trunk/ruthesis/latex/DEGREE-NAME-YEAR.tex $
%% $Id: DEGREE-NAME-YEAR.tex 360 2019-02-13 22:04:35Z foley $
%% This is a template LaTeX file for dissertations, theses, or reports at Reykjavík University
%% 
%% Comments and questions can be sent to the RU LaTeX group (latex AT list.ru.is) 
%% ---------------------------------------------------------------

%% METHOD:
%% 0) Read ruthesis/thesis-instructions.pdf
%%    If it is missing, goto https://repository.cs.ru.is/svn/thesis-template/trunk/ruthesis/thesis-instructions.pdf
%% 0.2) Subscribe to the announcements email list at
%%    https://list.ru.is/mailman/listinfo/latex-announcements
%% 1 LaTeX instructions.tex or goto http://afs.rnd.ru.is/project/thesis-template/trunk/ruthesis/latex/instructions.pdf
%% 2) Copy the template files (or unzip) to your working area
%% 3) Rename this file (if needed) with your information e.g. MSC-FOLEY-2007.tex
%% 4) Modify this file to fit your needs (please follow all comments below in the text)
%% 5) For making bibliographies, run "biber".  You can also change
%%    this back to "bibtex".  See below in "Bibliography options".

%%%%%%% CHOOSE ONE OF THESE %%%%%%%%%%%%%%%
%% projectreport: Project report (CS)
%% bachelors: Bachelor of Science thesis
%% masters: Master of Science thesis
%% doctorate: Doctor of Philosophy dissertation
%
%%%%%%% CHOOSE ONE OF THESE %%%%%%
%% 
%% draft: speed up processing by skipping graphics and adding useful
%%     information for editing.  Also sets spacing to double so that it is easier to
%%     write editing marks on paper copy.
%% proof:  proofreading version (final formatting with warnings)
%% final: generate document for submission, removing FIXMEs, and
%%     other markup.  Throw error if any fatal FIXMEs still in document.
%%
%%%%%%% CHOOSE ONE OF THESE IF APPLICABLE %%%%%%
%%
%% deptsse: School of Science and Engineering
%% deptscs: School of Computer Science
%%
%%%%%%%% CHOOSE ANY COMBINATION OF THESE %%%%%%%%%%%%
%%
%% forcegraphics: force graphics, etc. to be included, even in draft mode
%% debug:  writes more messages to the log file, adds debugging output 
%%     and sizing boxes
%% icelandic: thesis is in Icelandic
%% oldstyle:  use the PhD headers and footers from the old CS template
%% online: for online versions (skip blank pages)
\documentclass[online,masters,deptsse,forcegraphics,draft]{ruthesis}

%%%%%%%%%%%%%%%%%%%% TeXStudio Magic Comments %%%%%%%%%%%%%%%%%%%%%
%% These comments that start with "!TeX" modify the way TeXStudio works
%% For details see http://texstudio.sourceforge.neit/manual/current/usermanual_en.html   Section 4.10
%%
%% What encoding is the file in?
% !TeX encoding = UTF-8
%% What language should it be spellchecked?
% !TeX spellcheck = en_US
%% What program should I compile this document with?
% !TeX program = xelatex

%%%%%%%%%%%%%%%%%%%% Bibliography options %%%%%%%%%%%%%%%%%%%%%
%% We suggest switching from bibtex to biblatex/biber because it is better able
%% to deal with Icelandic characters and other bibliography issues
%% As long as you use biblatex instead of bibtex by itself, it will at least
%%  generate a document without errors.
%% !!!If you are using TeXStudio, don't forget to update the bibliography setting!!!
\usepackage[backend=biber,bibencoding=utf8,style=ieee]{biblatex}
%\DeclareLanguageMapping{american}{american-apa}  
% need to declare mapping for style=apa to alphabetize properly
% If you set backend=bibtex, it will use bibtex for processing (old way)
%    this can work with Icelandic characters, but you may get weird results.
%    bibtex does not know how to sort Þ and ð
% if you set backend=biber, you can use UTF8 characters such as Þ and
%     ð  but you will have to remember to switch from using bibtex to 
%     biber in your client
% If you use JabRef, make sure the file is encoded in UTF-8 which is
%    not the default.

%% This tells TeXStudio to use biber
% !TeX TXS-program:bibliography = txs:///biber
%% This also sets the bibliography program for TeXShop and TeXWorks
% !BIB program = biber

% Where is your reference library?

\addbibresource{zot-references.bib}

%%%%%%%%%%%%%%%%%%% CUSTOMIZATIONS %%%%%%%%%%%%%%%%%%%%%%%%%%%%%
%% It is not recommended that you customize this file nor
%% ruthesis.cls.  Just fill in the necessary fields.  You should put
%% your macros and packages into a separate file so that it is easier
%% to use updates to the template.  The custom.sty file was created
%% for this reason.  We load this much later so that it can overrite
%% any existing settings
\IfFileExists{custom.sty}{\usepackage{custom}}{}


%%%%%%%%%%%%%%% INFORMATION %%%%%%%%%%%%%%%%%%5
%% University information must be multilingual to deal with the
%%  required cover pages and abstract on thesis
%% NOTE: This may not be required for other reports!!!

%% Babel Icelandic macros are setup  on RedHat at
%% /usr/share/texlive/texmf-dist/tex/generic/babel/icelandic.sty
%% /usr/share/texlive/texmf-dist/tex/generic/babel-icelandic/icelandic.ldf


%% Multilingual macros
%\newML{macroname}{englishword}{icelandicword}
%  creates \macronameML
%    \MLmacroname[english] - returns the english word
%    \MLmacroname[icelandic] - returns the icelandic word
%    \MLmacroname  - uses the current language setting
% Some useful ones have already been defined, but can be redefined
%% Predefined: \MLIceland \MLReykjavikUniversity \MLUniversityIceland

%% What institute?  Default is RU.
%\setInstitution{\MLReykjavikUniversity}
% \newML{InstitutionAddress}{Menntavegur 1\\101 Reykjavík, Iceland}
% {Menntavegi 1\\101 Reykjavík, Ísland}
% \setInstitutionAddress{\MLInstitutionAddress}
% \newML{Tel}{Tel.}{Sími}
% \setInstitutionPhone{\MLTel{} +354 599 6200\\
% Fax +354 599 6201}
% \setInstitutionURL{www.ru.is}


%% ONLY SET DEPARTMENT IF YOU HAVE NOT USED THE deptsse or deptscs OPTION!
%% Department and degree program
%\newML{ND}{New Department}{Nytt deild}
%\setSchool{\MLND}

%% Set your program of study
\newML{program}{Mechatronics}{Hátækniverkfræði}
\program{\MLprogram}

%% Degree long name.  If not already defined, you can create a macro
%\newML{DEGREE}{English Degree Name}{Icelandic Degree Name}
%% Default is set based upon doctorate vs masters option
%% Predefined: \MLMSc \MLPhd
%\setDegreelong{\MLMSc}

%% Degree abb, change if default is not right
%% Default is set based upon doctorate vs masters option
%\degreeabbrv{Sc.D.} 

%\setFrontLogo{reyst-logo}
%% Use this if you need a different front logo on the first page
%% e.g. reyst-logo

%% Date in english and icelandic
%% NOTE: THIS IS THE DATE OF THE SUPERVISOR'S SIGNATURE!!!!!!
%% Predefined: \MLjan, \MLfeb, \MLmar, ... \MLdec
%\whensigned{day}{month}{year} %day is only used on some formats, but you must put something.
\whensigned{10}{June}{2021}

%% Title first in English then Icelandic
%% You need to put both a normal case and ALL CAPS version into the macros.
%%
\newML{Title}{Gathering cetacean vocalization using a self-sufficient buoy}{Hljóðsöfnun á hvölum með notkun sjálfbærri bauju sem framleiðir sitt eigið rafmagn}
\newML{TITLE}{REYKAJVÍK UNIVERSITY PROJECT REPORT, THESIS, AND DISSERTATION TEMPLATE}{TITLL VERKEFNIS med ÞÖÆÉÍÓ}
%%
\setTitle{\MLTitle}{\MLTITLE}
%% ***** Special Titles ******
%% If the title must be formatted specifically for the cover page or internal pages
%% (typically via line-breaks using the \newline command) then the following commands must be used 
%%
%\setTitleCover{\MLTITLE}
%% These two for the internal cover pages, usually not needed
%\newML{TitleInternal}{Internal Title}{Icelandic Internal Title}
%\setTitleInternal{\MLTitleInternal}

%% Author name (should be the same in any language, if not use \newML)
%% If you are writing a Project report with multiple authors, separate them with \\:
%% To keep the names typeset together, you want to use non-breaking spaces: ~
%\author{Firstname1~Lastname1\\Firstname2~Lastname2}
\author{Sigurbergur~Magnússon}

%% If the name must be formatted specifically for the signature page
%% (typically via line-breaks) then the following command must be used 
%\setAuthorSignature{Student\\Name}
%% This macro adjusts the author name in the headers of the oldstyle formatting
%\setAuthorHeader{StudentLast}

%%% TODO:  Move the bachelor's form separately -- it confuses people. --foley
%%%%%%%%%%%%%%%%%%%%%%%%%% Project Report or Bachelor's Only!!! %%%%%%%%%%%%%%%%%%%%%%%%%%%%%%%%%%%%%%%%%%%
\setCourse{VT LOK 1012}

%%%%%%%%%%%%%%%%%%%%%%%%%% Bachelors Only!!! %%%%%%%%%%%%%%%%%%%%%%%%%%%%%%%%%%%%%%%%%%%
\setID{200594--2079}%kennitala
\setSemester{2016--1}
\setShortSignedDate{1.1.2016}

% \setOrganization{Marel ehf.\\Austurhrauni 9\\210 Garðabær}
% \setSubProgram{Tæknifræði}

%% If the thesis is confidential, uncomment this with the date it can be released
%\setClosedDistribution{10.1.2016}%

%% Put your keywords here in English, then Icelandic.  Separate them with commas.
\newML{keywords}{Cetacean vocalization, Digital audio recording}{Hvala hljóð, Stafræn hljóðupptaka}
\setKeywords{\MLkeywords}

%%%%%%%%%%%%%%%%%%%%%%%%%%% Masters Only!! %%%%%%%%%%%%%%%%%%%%%%%%%%%%%%%%%%%%%%%%%%%%
%% How many credits (ECTS) on Master's degree
%% Usually 30 or 60
\ects{30}

%%%%%%%%%%%%%%%%%%%%%%%%%%% Doctorate Only!! %%%%%%%%%%%%%%%%%%%%%%%%%%%%%%%%%%%%%%%%%%
%% Some Computer Science Thesis have an ISSN number.
%% Most other documents do not.
%\bookidnumber{ISSN: 1670-8539} 
%% ID numbers are optional, but nice for sorting in libraries

%% International Standard Book Number (ISBN)
%% This is what most people should use if the thesis is being published.

%% International Standard Serial Number (ISSN)
%% This is usually only for a PhD dissertation as part of a series when published
%%   Computer Science: 1670-8539 

%% Additional degrees?  (optional, usually not needed)
%\adddegree{(list of degrees in appendix)}{(sjá lista yfir prófgraður í viðauka)}
%%%%%%%%%%%%%%%%%%%%%%%%%%%%%%%%%%%%%%%%%%%%%%%%%%%%%%%%%%%%%%%%%%%%%%%%%%%%%%%%%%%%%%%%


%% List the entire committee.  Each member has a name (degree should be omitted, unless it is not PhD),
%% Supervisor(s) must appear first
%% On a Bachelors, there is usually only one supervisor and one examiner.

%% Format for each entry:
%%  \personinfo{Name}{Role}{Job Title}{Company/institution}{Country}
%% Predefined macros: \MLSupervisor \MLSupervisors \MLExaminer \MLExaminers

%% Change these to singular/plural as needed.
%% Just uncomment and change the plurality of the macro.
%\setSupervisorHeading{\MLSupervisors}
%\setExaminerHeading{\MLExaminer}

%% Predefined macros:
%% \MLSeniorProfessor \MLProfessor \MLAssociateProfessor \MLAdjunctProfessor \MLEmeritusProfessor \Iceland
%% \MLReykjavikUniversity \MLUniversityIceland

%% Bachelors: primary advisor (Umsjónarkennari), ONLY ONE!
%% All others: As many as you want
\supervisors{
  \personinfo{Baldur Þorgilsson}{\MLSupervisor}{\MLProfessor}{\MLReykjavikUniversity}{\MLIceland}
%  \personinfo{Helpful A. Teacher}{Co-advisor}{\MLAssistantProfessor}{\MLUniversityIceland}{\MLIceland}
%  \personinfo{Ian M. Great}{Co-advisor}{\MLProfessor}{Hochschule Düsseldorf}{Germany}
}

%% Bachelors: secondary advisor (Leiðbeinandi), ONLY ONE
%% All others: As many as you want
\examiners{
  \personinfo{\fxfatal{Bæta við prófdómara}}{Associate Professor}{Massachusetts Institute of Technology}{USA}

}

%% An abstract is required to be in both Icelandic and English for most degrees.
%% It is considered good form to limit the abstract to a single paragraph in each language,
%%   at 300 words.  Refer to your degree's instructions.
%% Note: Icelandic quotation marks cannot be typeset using "` and "'.  You should use \enquote{}
%% this is probably due to interactions with the MultiLingual macros.
%% TODO: turn this into more sensible macros to avoid confusion --foley
\newML{AbstractText}{

With ever increasing pressure on wildlife caused by humans, it becomes ever so more important to monitor the changes that might occur in the behavior and habitats of affected animals.
As well as identifying the factors that causes disturbances to the animal, to minimize or eliminate them.
Therefore creating new or improving old methods for researching and monitoring the impacts on the animals, which is crucial for the preservation of wildlife.
This thesis is a part of a larger project that proposes the design of a self-powering buoy with vocalization recording capabilities and can transfer that data onshore to researchers and will focus specifically focus on the vocalization recording aspect.
The proposed recording device consists of the Teensy 3.5, Aquarian H1a hydrophone which can record vocalization signals up to 100kHz.
The device has a low power consumption of 0.415W and a low cost of 190\$.
The device uses a programmable delay block to trigger the built-in analog to digital converter to take measurements and direct memory access is used to transfer the converted digital value to a buffer which later writes the data onto a SD card.
Currently, the device can sample signals at 16 bits resolution can record up to 300ksps which should allow it to record signals up to 150kHz.


}  
% ipsum generaes text text
{

Með auknum áhrifum mannfólks á vilt dýr, þá verður eykst mikilvægi þess að fylgjast breytingum í hegðun og búsvæðum dýrana.
Einnig þarf þá að reyna að finna þá áhrifavalda sem trufla dýrin, til að annaðhvort minnka áhrifa þessa eða eyða þeim algjörglega.
Þess vegna er mikilvægt að búa til eða bæta gamlar aðferðir til að fylgjast með og rannsaka þau áhrif sem mannfólk hefur á dýrin, í þágu varðveislu dýra.
Þessi ritgerð fjallar um mismunandi möguleika á óvirku hljóðvistareftirliti fyrir rannsóknir á hvölum.
Þessi ritgerð er hluti af stærra verkefni þar sem tillagan er að nota sjóbauja sem getur framleitt rafmagn fyrir orkulítlum raftækjum sérhannaðar í hvalahljóðs upptöku og getur síðan send þau gögn á land, en ritgerðin mun fjalla sérstaklega um hljóðupptöku þátt sjóbaujunar.
Tillagan af því tæki er að nota Teensy 3.5, og Aquarian H1a vatns hljóðnema sem getur tekið upp hljóðmerki upp að 100kHz.
Tækið notar lítið rafmagn eða um 0.415W og kostar lítið eða 190\$.
Tækið notar forritanlega bið blokk til þessa að kveikja á hljóðrænum-stafrænum umbreyti til að taka mæligildi og notar síðan beinan minnisaðgang til þess að geyma þau gildi í biðminni sem seinna skrifar gögnin á SD kort.
Eins og er getur tækið tekið upp merki með 16bita upplausn á 300ksps, sem gerir því fræðilega kleyft að taka upp merki allt að 150kHz.
Þessi ritgerð mun sýna fram á að tækið sem lagt er til er hægt að nota í að hjálpa hvala rannsökurum við rannsóknir og getur hjálpað í þágu náttúrverndar.

} % Icelandic abstract goes here
\setAbstract{\MLAbstractText}


%%%%%%%%%%%%%%INDEX SETUP %%%%%%%%%%%%%%%%%%%%%%%%%%%%%%%%%%%%%%%%%%%%%%%%%%%%
%% Indexes, and other auto-generated material
%% The Memoir package (which we use) automatically generates the index
%% See section 17.2 on page 302 of the guide
%% http://texdoc.net/texmf-dist/doc/latex/memoir/memman.pdf
%% This means you have to run "makeindex DEGREE-NAME-YEAR"
%% !!!Do not load any of the index packages, they cause problems with Memoir!!!
%% !!!You have been warned!!!
%% Note that memoir changes the [] options to only be for filenames, not other options!
\makeindex{}
\indexintoc{}

%% For abbreviations, you may want to try
%% Watch out though, each new index writes another external file and 
%% latex can only write a limited number of them
%%\usepackage[intoc]{nomencl} % intoc: In Table of Contents
%% remember to run:
%% makeindex filename.nlo  -s nomencl.ist -o filename.nls

\finalifforcegraphics{hyperref} %hyperlinks even in draft mode
\usepackage[hidelinks]{hyperref} 
%% !!!Must be the last package loaded except otherwise mentioned!!!!
%% \usepackage{hypcap}  %% puts link at top of figure, must be after hyperref

%%%%%%%%%%%%%%%%%%%%%%%%%%%%%%%%%%%%%%%%%%%%%%%%%%%%%%%%%%%%%%%%%%%%%%%%%%
%%%%%%%%%%%%%%%%%%%%%%% DOCUMENT START %%%%%%%%%%%%%%%%%%%%%%%%%%%%%%%%%%%
\begin{document}
%% Some elements have different names on the RU Masters rules
%% They will be annotated with RUM: "name"
\frontmatter{} % setup formatting at beginning

%\frontcover{}%%If you want to see what it looks like with the printed cover
%% TODO:  link to fill-in PDF file on RU website

\frontrequiredpages{}%% the various signaturepages and abstract
%%% WARNING:  if you get an error on the previous line, it is probably because
%%% you put a bad macro or something strange in a title, author, or abstract.

\ifdraft{\coverchapter{Important!!!  Read the Instructions!!!} If you
  have not already done so, \LaTeX{} the \path{instructions.tex} to
  learn how to setup your document and use some of the features.  You
  can see a (somewhat recent) rendered PDF of the instructions included in this folder at \path{instructions-publish.pdf}.
  There is also more information on working with \LaTeX{} at
  \url{http://samvinna.ru.is/project/htgaru/how-to-get-around-projects-publish.pdf}.
  This includes common problems and fixes.

  This page will disappear in anything other than draft mode.}{}



%% Dedication is optional, comment out if it is absent
%% RUM: Not mentioned
\begin{dedications}
  I dedicate this to my family and friends who have helped me on my journey.
\end{dedications}

\enableindents{}% turn on/off paragraph indents
% RUM: "Acknowledgements (optional)"
%\coverchapter{Acknowledgements} 
%\begin{quotation}
%So long, and thanks for all the fish.
%\end{quotation}\sourceatright{Douglas Adams\cite{adams84fish}}
%\vspace{\baselineskip}

%\draftnote{Acknowledgements are optional; comment this chapter out if they are absent
%  Note that it is important to acknowledge any funding that helped in the work}

%This work was funded by \the\year~RANNIS grant ``Survey of man-eating Minke whales'' 1415550.
%Additional equipment was generously donated by the Icelandic Tourism Board.

%\coverchapter{Preface}
% RUM: "Preface (optional)"
%This dissertation is original work by the author, Firstname~Lastname.
%Portions of the introductory text are used with permission from
%Student et al.\cite{student2015awesome} of which I am an author.

  
%\draftnote{The preface is an optional element
%  explaining a little who performed what work.  See
%  \url{https://www.grad.ubc.ca/sites/default/files/materials/thesis_sample_prefaces.pdf}
%  for suggestions.
  
%  List of publications as part of the preface is
%  optional unless elements of the work have already been published.
%  It should be a comprehensive list of all publications in which
%  material in the thesis has appeared, preferably with references to
%  sections as appropriate.  This is also a good place to state
%  contribution of student and contribution of others to the work
%  represented in the thesis.}

%\coverchapter{Publications}
%% RUM: Not mentioned, this was found in the CS thesis template.  
%% Maybe more applicable to PhD dissertations?
%%% Probably a duplication from before Preface became standard.

\starttables{}% setup formatting
%% TOC, list of figures and list of tables are required
\tableofcontents{}\clearpage%%RUM: "Table of contents"
\listoffigures{}\clearpage%%RUM: "List of figures"
\listoftables{}\clearpage%%RUM: "List of tables"

%\coverchapter{List of drawings and enclosed material}
%RUM: "List of drawings and enclosed material, e.g. CD(as appropriate)"

\listoffixmes{}
% if using fixme package, lists what needs to be done


%% The list of abbreviations is an example of a special list
%% Other lists may be added, such as lists of algorithms, symbols, theorems, etc.
%% IN CS PhD, this is sometimes centered.

\coverchapter{List of Abbreviations}%%RUM: Not mentioned

\begin{tabular}{ll}

AAM & Active acoustic monitoring\\
ADC &Analog to digital converter\\
ADCK &ADC clock\\
DAC & Digital to anlaog converter\\
dB  & Decibels\\
DCV & Direct current voltage\\
DMA & Direct Memory Access\\
DMON & Digital acoustic monitoring\\
DPS & Digital signal processor \\
FFT & Fast Fourier transform \\
GPIO & General Purpose Input/Output\\
GPS & Global Positioning System \\
I/O & Input/output\\
IDE & Integrated development environment\\
ISR & Interrupt service routine\\
LSB & Least significant bit\\
kB & Kilobytes\\
Km &Kilometer\\
LFDCS & Low‐frequency detection and classification system\\
MMOs & Marine mammal observer\\
MSc &Masters of Science\\
OCRR & Open circuit receiving response\\
PAM & Passive acoustic monitoring\\
PDB & Programmable delay block\\
PWM & Pulse width modulation\\
RC & Resistor and capacitor\\
RMS & Root mean square\\
SIL & Sound intensity level\\
SNR & Signal to noise ratio \\
SPS & Samples per second\\
TVR & Transmitting voltage response\\


% PhD &Doctor of Philosophy\\
\end{tabular}

\coverchapter{List of Symbols}%%RUM: Not mentioned

\begin{tabular}{lll}
Symbol &Description &Value/Units\\
$E$ &Energy &\si{\joule}\\

$m$ &Mass &\si{gram}\\
$c$ &Speed of Light &\SI{2.99E8}{\meter\per\second\square}\\
$Hz$ & Frequency & Hz\\
$Ah$ & Ampere hour & 3600 coulombs\\
$F_c$ & Cutoff frequency & Hz\\

\end{tabular}

%% This command prepares for the actual text, e.g. by 
%% calling \mainmatter{}
\starttext{}

%% ---------------------------------------------------------------
%% From this point on, it is standard Latex, except the very end.
%% This is a "report"-based template, so the top-level heading 
%% is \chapter{}

%% WARNING: Make sure that all of these files (and any new ones)
%% are UTF-8 otherwise you will get weird encoding errors.
\part{The First Part} % Parts optional but useful in longer documents
Introduction to the first part.
%% The default division is IMRAD, you may want to divide differently
%% See the introduction for guidance.




\chapter{Introduction\label{cha:introduction}}
%% \ifdraft only shows the text in the first argument if you are in draft mode.
%% These directions will disappear in other modes.

% \ifdraft{State the objectives of the exercise. Ask yourself:
%   \underline{Why} did I design/create the item? What did I aim to
%   achieve? What is the problem I am trying to solve?  How is my
%   solution interesting or novel?}{}

% The object of the project is to be able to monitor cetacean traffic around Iceland.
% This could help tourism companies relating to whale watching. 
% Could also be extended to whale researchers.


Humans have caused increased pressure on animals whether that be land or sea creatures.
Keeping track of the creatures is vital to ensure their health and survival for the future.
This can be done by population assessments, which can show if the specific species is in an upsurge or declining.
For fish and cetaceans, there are several ways to achieve this for example the animal can be tracked via Global Positioning System (GPS) tracker or the subject is visually sighted by the use of a boat or aircraft.
These methods are however time consuming and require a lot of man power.
There is however a relatively new method of using  acoustic surveys in order to monitor cetaceans.
These can be done in several ways.
One method involves dragging an array of hydrophones and record the vocalization data.
The second is to mount hydrophones into the bow of the ship and record the vocalization data.
These method however can not be used when recording low frequency sounds, due to noise created by the ship and water flow.
The third method, and the one this thesis will focus on is a passive buoy sound gathering.
Where a hydrophone and a recording device are attached to a buoy that moored in place and sets on recording passively that location without the need of external help.

The buoy must be self sufficient and able to stay out at sea for long periods at a time to serve its function.
This means that all the electronics on board will need to be powered by the buoy.
It also has to be able to record cetacean vocalization and transmit the data onshore.
So that marine researchers can study it or even whale watching companies in the tourism industry can play cetacean vocalizations in the fjords.
This means the passive sound gathering buoy can be separated into three different projects, power production, sound gathering and data transmission onshore.
This thesis will focus on vocalization gathering part of the buoy.

This function will be implemented using a Teensy 3.5 microcontroller.
A hydrophone will be used in order to sense the sound waves generated by the cetacean vocalization which have an extremely wide frequency range of a few Hz all the way to echolocation signals which are hundreds of thousands of Hz, this project will have therefore gather signals in the range of 10Hz to 100kHz \textbf{BÆTA VIÐ MEIRI STATS UM VEERKEFNIÐ}.
The electrical circuit will consist of a preamplifier, filter, analog to digital converter and SD card for data logging.
The circuit will refine the signal to be readable for the microcontroller. 
The data gathered by the signal will then be sent onshore for further processing or utilization.


\section{Project Goals}

The objective of the project is to create a relatively small, low cost and low maintenance device capable of recording cetacean vocalization and transmitting the data onshore.
The design criteria for the project is as follows.
The total power consumption to the system should be less than 10 watts. 
The device should be relatively small and deployable by one person.
The device should last 6 months at a time.
Be able to gather data of signal ranging from 10Hz to 100kHz.


\section{Background}

Considerable cetacean preservation efforts have been carried out for the past decades having started in 1946 with a International whale committee.
Which is a global organization with the goal of conserving and managing whales and it currently has 88 countries signed to the committee \cite{noauthor_iwc_nodate}.

Many methods have been used to monitor the population of cetaceans.
These methods range from very active hands-on surveys where Marine Mammal Observers (MMOs) capture, examine, mark and then let the animal go to be recaptured in the future.
To a more passive acoustical surveys where hydrophones are utilized to listen in on cetaceans.

%https://www.sciencedirect.com/science/article/pii/S0003347216301452#:~:text=We%20describe%20several%20methods%20developed,high%2Dresolution%20acoustic%20recording%20tags.

\subsection{Visual surveys}% and photogrammetry 

Visual surveys are generally carried out by the use of helicopters, airplanes or boats.
Trained researchers using high power binoculars search for cetaceans breaking the surface of the ocean. 
Once the cetacean is sighted it is categorised accordingly with regards to the study.
Immense data can be extrapolated from such research such as the location of the sighting, the species sighted, group size to name a few\cite{campbell_inter-annual_2015}.
How ever there are several factors that might limit or stop a visual survey such as sea conditions, visibility, the behaviour of animals and the animal could lie just beneath the surface out of MMOs sight.

\subsection{Hands-on surveys}

This method utilizes animals that are caught and released, animals that are being cared for and animals that have become incapacitated by beaching. 
Animals that are caught are identified, which can be done by taking photographs of markings or identifying features 
%that could later be used if the animal is recaptured used in order to identify it,
which is known as photo-identification\cite{booth_methods_2020}.
This approach has been utilized in order to gain a further understanding on population and health of cetaceans.

Individual tracking surveys are similar in the fact that the animal is captured and released.
However instead of taking photographs of it before releasing it, the animal would have a GPS tracker attached to it. 
Which has been utilized for research into acquiring data regarding behaviour and responses of disturbance sources. 
As well as data on the animals travel and habitat patterns. 
Depending on what the researcher wants to get data on will dictate on whether or not the GPS tracker is added on to the animal\cite{booth_methods_2020}.

\subsection{Acoustic surveys}

Methods previously mentioned rely heavily on favorable conditions regarding sea conditions and visibility since the MMOs need to be able to spot the cetaceans.
There are different methods to conduct the surveys, either active or passive.
Active acoustic monitoring (AAM) includes systems such as fish sonars and echo sounders. 
Cetaceans are detected with target reflection instead of vocalization \cite{pyc_evaluation_2015}.
Passive acoustic monitoring (PAM) relies on the use of hydrophones, where cetacean vocalizations are recorded and studied.
There are several methods to choose from when conducting a PAM.
One involves towing an array of hydrophones behind a vessel at sea such as ship and more recently autonomous platforms platforms\cite{baumgartner_diel_2008}.
Which still requires favourable sea conditions and the use of active MMOs on board.
Another is to mount hydrophones in the bow of the ship.
This method is however limited frequency range due to the noise created by the ship bow and water flow\cite{rankin_acoustic_2008}.
The third is to have a fixed device, that has hydrophones and is able to either store sound data on the device itself or transmit the data directly on shore to researchers.
These devices are generally capable of long term unmanned monitoring and can be a quite cost effective alternative.
All three methods can eliminate most if not all of the previously described problems that can occur with visual surveys.



\subsubsection{$\mu$RUDAr-mk2}

Devices such as the $\mu$RUDAR-mk2 as seen in \textit{Figure~\ref{fig:uRUDAR}}, is a product from Cetacean research technology that offers a remote fixed monitoring system.
It has the ability to remote autonomous recording.
WiFi recording control and data transmission capability.
The device is able to record up to 24-bit/96kHz and 45kHz bandwidth and up to 16.5 days of continuous recording time\cite{computing_microrudar_nodate}.
This device however relies on battery power and therefore has limitations on deployment time.

\begin{figure}[h]
    \centering
    \includegraphics[width=0.70\textwidth]{graphics/uRUDAR-mk2.jpg}
    \caption{uRUDAR-mk2 fixed monitoring device\cite{computing_microrudar_nodate}}
    \label{fig:uRUDAR}
\end{figure}

\subsubsection{RUDAR (Remoter Underwater Digital Acoustic Recorder)}
Another recording device from Cetacean research technology is the RUDAR (Remoter Underwater Digital Acoustic Recorder), seen in \textit{Figure~\ref{fig:Rudar}}. 
It is a autonomous recording device that is small enough to be hand deployed from a small boat.
The recording system uses the ST400 mobile data recorder and sound level monitor. 
The system has a working depth of 1.5-3.5km, is able to record from 4 hydrophones at 24-bit resolution.
The data is written to internal hard drives and is able to record 2 independent schemes and sample rates at the same time \cite{cetacean_research_technology_rudar_2021}.

\begin{figure}[h]
    \centering
    \includegraphics[width=0.70\textwidth]{graphics/Rudar.jpg}
    \caption{RUDAR recording device \cite{cetacean_research_technology_rudar_2021}}
    \label{fig:Rudar}
\end{figure}

\fxfatal{TALA KANSKI UM F-pod hér \cite{noauthor_f-pod_nodate}}

\subsubsection{Persistent near real‐time passive acoustic monitoring for baleen whales from a moored buoy.}
A system has been developed for the United States Coast Guard that is capable of long term remote deployment. 
The system consist of a moored buoy that can provide data collection and transmission.
It has passive acoustic instruments such as digital acoustic monitoring(DMON) and a low-frequency detection and classification(LFDCS) firmware.
The system has three hydrophones and a a programmable Texas Instruments TMS320C55 digital signal processor (DPS) as well as GPS.
The firmware is used to build a spectogram of the recorded sounds when the mooring is recovered.
An example of which can be seen \textit{Figure \ref{fig:SpectoExamp}}.
It then classifies the sound calls by comparing attributes of the pitch track to known call types. 
The mooring hardware of the surface buoy is used for power delivery as well as data transmission. 
The system has an internal battery capacity of 450 Ah. 
%The audio is recorded with a sample frequency of 2kHz. 
It is designed to operate for a period of 1 year at a time has a maximum data transfer rate of 8Kb per hour through Iridium global communication system \cite{baumgartner_persistent_2019}.

\begin{figure}[h]
    \centering
    \includegraphics[width=0.60\textwidth]{graphics/spectogram.png}
    \caption{Spectogram that can be produced once the mooring is recovered.
    (a) In near-real time detection information (b) spectogram for a the given time period seen in (a) at 2000Hz sampling rate (c) addition of annottations of sound source to spectogram seen in (b)\cite{baumgartner_persistent_2019}}
    \label{fig:SpectoExamp}
\end{figure}

The setup of the of the system can be seen in \textit{Figure~\ref{fig:DMON/LFDCS}} from the moored surface buoy at the top, to the monitoring device at the bottom of the ocean.

\begin{figure}[h]
    \centering
    \includegraphics[width=0.70\textwidth]{graphics/DMONbuoy.jpg}
    \caption{A moored buoy using DMON/LFDCS systems, designed for 1 year deployments\cite{baumgartner_persistent_2019}}
    \label{fig:DMON/LFDCS}
\end{figure}




%%% Local Variables: 
%%% mode: latex
%%% TeX-master: "DEGREE-NAME-YEAR"
%%% End: 
%%RUM: Introduction
\chapter{Methods}

%\textbf{Tala um raunverulega ásæðu af hverju ég vel þessa íhluti!!!!!!}

\section{Hydrophone}\label{sec:AquarionHydro}

%\textbf{TALA UM AÐRA TÝPU SEM HEFÐI GETAÐ KOMIÐ TIL GREINA
%https://www.nauta-rcs.it/English/Instruments/Hydrophones/CetaceanResearchTech/C55/C55.html}

The first thing that needed to be determined was the hydrophone used.
Generally hydrophones are sensitive piezoelectric sensors that can detect small changes in pressure and convert that to an electrical signal.
This project will use the Aquarian audio H1a hydrophone, which was chosen for its low cost and good sensitivity, it was also readily available at the Reykjavik University lab.

\begin{table}[h]\caption{Important specifications of the Aquarian H1a hydrophone.\cite{noauthor_aquarian_nodate}}.\label{Tab:Aquarian}
\begin{tabular}{l|l}
Sensitivity     & -190dB re: 1V/µPa (+/- 4dB 20Hz-4KHz) \\\hline
Useful range    & \begin{tabular}[c]{@{}l@{}}100KHz (not measured above 100KHz, approximate sensitivity\\  @100KHz = -220dB re: 1V/µPa)\end{tabular} \\\hline
Capacitance     & 25nF \\ \hline
Operating depth & \textless{}80meters\\ \hline
Cost            &  159\$ \\ \hline
\end{tabular}
\end{table}

There is no preamplifier or impedance buffer circuit within the Aquarian H1a hydrophone, so the circuit needs to amplify the output signal of the hydrophone.
The gain of the circuit depends on what the intended use of the hydrophone will be.
In this case for cetaceans and large aquatic wildlife, Aquarian recommends gain of 40 - 50 dB and for very distant sounds 60dB.

\clearpage

\begin{equation}
    F_c = \frac{1}{ 0.000000157 R}  
\label{eq:fchydro}
\end{equation}

The Frequency response of the Aquarian H1a hydrophone can be calculated using \textit{Equation~\ref{eq:fchydro}} \cite{noauthor_aquarian_nodate}. 
Which is just \textit{Equation~\ref{eq:FC}} where the capacitance of the hydrophone has been multiplied with $2\pi$.


\subsection{Microcontroller og ADC}

\subsubsection{Microcontroller}
The Teensy3.5 was ultimately chosen for the microcontroller of the system.
This decision was made for several reasons.
The Teensy3.5 has two built-in ADC, which means the system might be able to add another hydrophone to it for recording.
Both ADCs have a maximum resolution of 16bits with a voltage range of 0 - 3.3V, or $\frac{3.3}{2^{16}} \approx 0.00005V \approx 50\mu V$ step size.
The Teensy also consumes little power, when running with no peripherals active the processor is powered roughly by 50mA at 5V or 0.25W.
Which is under the requirements set at the beginning of the project.
As well as it was readily available at the Reykjavik University lab.

\subsubsection{Operational amplifier}
%\textbf{Mögulega tala um í kafla 2 hvað mikilvægustu þáttir opampa fyrir þetta verkefni var}

The operational amplifier that was chosen for this project was the OPA1644 from Texas Instruments.
It was chosen its advertised superior sound quality and low noise spectral density of $\frac{5.1nV}{\sqrt{Hz}}$ at 1kHz.
%\begin{figure}[h]
%    \centering
%    \includegraphics[width=0.7\textwidth]{graphics/noiseDensvsFreq.png}
%    \caption{\textbf{BÆTA}}
%    \label{fig:noiseDensvsFreq}
%\end{figure}
%Which means 
%https://www.designnews.com/what-nvvhz-noise
%\textbf{KYNNA Sér SNR}
%$$\sqrt{100*10^3Hz - 10Hz} = 3\sqrt{11110}Hz$$
%$$3\sqrt{11110}Hz*5.1*10^{-9}V = 1.6*10^{-6}V$$
%$$ 1.6*10^{-6}V * 100 = 1.6*10^{-4}$$
%Using a 1V output (0dBV) from the op amp the signal to noise ratio is:
%$$SNR = 20log_{10}(\frac{1V}{1.6*10^{-4}V}) \approx 76dB$$
It has low distortion of 0.00005\% at 1 kHz and a high slew rate of $\frac{20V}{\mu s}$.
As well as having a supply range that fits within the voltage range ($\pm2.25~-~\pm18$) of the Teensy analog pins of 3.3V\cite{noauthor_opa164x_nodate}.
It also has up to 40dB gain at 100kHz as seen in \textit{Figure~\ref{fig:dbvsFreq}}.

\begin{figure}[h]
    \centering
    \includegraphics[width=0.7\textwidth]{graphics/dbVsfreq.png}
    \caption{Gain and phase shift vs frequency of the OPA1644\cite{noauthor_opa164x_nodate}}
    \label{fig:dbvsFreq}
\end{figure}


\vspace{4cm}


\section{Circuit}\label{sec:CircResult}

%From the project goals the circuit needs to be able to gather data from signals with frequencies between 10 - 100kHz. 
It was decided to collect signals ranging from 10 - 100kHz, the lower limit which was chosen after seeing \textit{Table~\ref{Tab:WhaleHz}} and the dominant frequencies, the higher limit was chosen because the project goal was to maximize the absolute limits of the device.
This could be significantly reduced since most dominant frequencies are lower than 30kHz however, there are vocalization that goes up to and over 100kHz so it is interesting to see if the Teensy was capable of recording these high-frequency signals.
The Aquarian H1a hydrophone datasheet specifies that for cetacean vocalization recording the gain of the preamplifier needs to be between 40 - 50 dB\cite{noauthor_aquarian_nodate}.
The circuit will there for consist of an active low pass filter and high pass filter, a scaling summing amplifier and an inverting op-amp.
It will then connect to the Teensy 3.5 built-in ADC.\\
\indent \textit{Figure \ref{fig:Opamp1}} shows the first part of the circuit.
The output of the hydrophone is first connected to a high pass filter, where the capacitance, 25nF of the hydrophone is used as the capacitor in the high pass filter.

\begin{figure}[h]
    \centering
    \includegraphics[width=0.70\textwidth]{graphics/OPamp1.png}
    \caption{The first part of the circuit where the output of the hydrophone first filtered using an active high pass filter and then connected to a operational amplifier and amplifying the signal by 20.}
    \label{fig:Opamp1}
\end{figure}

\vspace{4cm}

The desired value for the high pass filter was 10Hz and using \textit{Equation~\ref{eq:fchydro}} provided in the datasheet, the resistor value was estimated.   
$$F_c = \frac{1}{2\pi * 25nF * 636k\Omega} = 10Hz$$
Using standard resistor values, the closest resistor value is 620k which yields a cut off frequency of 10.26Hz.
The gain over the active high pass filter can be represented using \textit{Equation \ref{eq:DCGain}} and \textit{Equation \ref{eq:ActiveHighPass}}. 
Where $A_{V1} = 1 + \frac{19k\Omega}{1k\Omega} = 20$.
$$A_f = \frac{(1+\frac{19k}{1k})(\frac{f}{10.26Hz})}{\sqrt{1 + (\frac{f}{10.26Hz})^2}} \approx 20$$
Which is approximately equal to 20, as seen in \textit{Figure~\ref{fig:AVhighpass}} over the entire bandwidth.

\begin{figure}[h]
    \centering
    \includegraphics[width=0.5\textwidth]{graphics/Av_Highpass.png}
    \caption{The gain of the operational amplifier with the active high pass filter.}
    \label{fig:AVhighpass}
\end{figure}

\vspace{4cm}

\textit{Figure \ref{fig:Opamp2}} shows the active low pass filter configuration which has a cut-off frequency of $\approx$ 100kHz and a gain of 2.5.

\begin{figure}[h]
    \centering
    \includegraphics[width=0.7\textwidth]{graphics/OPamp2.png}
    \caption{The second operational amplifier, where the signal is first filtered by the active low pass filter as well as amplifying the signal.}
    \label{fig:Opamp2}
\end{figure}

The desired value for the high pass filter was 100kHz and using \textit{Equation~\ref{eq:FC}} and choosing a resistor value of 160$\Omega$, the capacitor value was estimated.   
$$F_c = \frac{1}{2\pi 9.9nF 160\Omega} = 100kHz$$
Using standard capacitor values, the closest capacitor value is 10nF which yields a cut off frequency of 99472Hz.
The gain over the active low pass filter can be represented using \textit{Equation \ref{eq:DCGain}} and \textit{Equation \ref{eq:ActiveLowPass}}. 
Where $A_{V1} = 1 + \frac{1.5k\Omega}{1k\Omega} = 2.5$.
$$A_f  = \frac{1 + \frac{1.5k\Omega}{1k\Omega}}{\sqrt{1 + (\frac{f}{99471.8Hz})^2}}$$
Which is approximately equal to 2.5 over the entire bandwidth. 
\textit{Figure~\ref{fig:AVlowpass}}.

\begin{figure}[h]
    \centering
    \includegraphics[width=0.6\textwidth]{graphics/Av_Lowpass.png}
    \caption{The gain of the operational amplifier with the active low pass filter.}
    \label{fig:AVlowpass}
\end{figure}

\textit{Figure~\ref{fig:Opamp3}} shows the scaling summing operational amplifier which is used to shift the input signal by 1.65Vdc. 


\begin{figure}[h]
    \centering
    \includegraphics[width=0.70\textwidth]{graphics/OPamp3.png}
    \caption{The scaling summing operational amplifier, where the signal is shifted by 1.65Vdc}
    \label{fig:Opamp3}
\end{figure}


Which is important since the analog pins of the Teensy 3.5 have a voltage range of 0-3.3V.
There for the signal needs to be shifted by 3.3V/2 = 1.65Vdc.
%To estimate the voltage output of the op amplifier, \textit{Equation \ref{eq:invertDCGain}} and \textit{Equation \ref{eq:ScalingGain}} was used.
%For an input signal of 10mV amplitude, of the scaling summing op amp would be \textbf{KANSKI EKKI HAFA ?!?!?!} 
%$$V_{out} = -180\Omega(\frac{1.65V}{180\Omega} + \frac{\pm 10mV}{90\Omega}) = \pm $$
A voltage divider was needed to determine the resistor values for the 1.65Vdc which would ultimately put some design restraints on R16 and R13 seen in \textit{Figure~\ref{fig:Opamp3}}.
Firstly it was decided to have $R10 = 100\Omega$  and $R11 = 225\Omega$ to start off the calculations.
From that R14 could be calculated as $180\Omega$.
$$3.3V = \frac{\frac{1}{\frac{1}{180\Omega}+\frac{1}{225\Omega}}}{100\Omega+\frac{1}{\frac{1}{180\Omega}+\frac{1}{225\Omega}}} = 1.65V$$
Which was changed to $160\Omega$ because of the addition of R16 and using Multisim it was determined that $160\Omega$ would yield the closest results to the 1.65Vdc.
The gain for the input signal from the low pass filter is found by using \textit{Equation \ref{eq:invertDCGain}}
$$A_{V3} = -\frac{180\Omega}{90\Omega} = -2$$
The operational amplifier is in an inverting configuration which turns positive voltage signals to negative.
This needs to be rectified since the Teensy 3.5 analog pins do not have a negative voltage range.
Which is done by another inverting operational amplifier the output of which connects to the A9 analog pin on the Teensy via a 330$\Omega$ resistor which limits the current to under 100mA, seen in \textit{Figure~\ref{fig:Opamp4}}.

\begin{figure}[h]
    \centering
    \includegraphics[width=0.70\textwidth]{graphics/OPamp4.png}
    \caption{The final operational amplifier, which is in an inverting configuration.}
    \label{fig:Opamp4}
\end{figure}

The gain of inverting amplifier is found using \textit{Equation \ref{eq:invertDCGain}}.
$$ A_{V4} = -\frac{1k \Omega}{1k \Omega} = -1$$
%The total circuit therefore consists of four operational amplifiers, high pass and low pass filter.
%\textbf{MÖGULEGA BÆTA JÖFNU 'I KAFLA 2}
The total gain of the circuit for the input signal is found by.

$$A_{tot} = A_1A_1A_2A_3A_4 = 20*2.5*-2*-1 = 100$$

Finding the total dB gain of the circuit can be found by \textit{Equation \ref{eq:DBGAIN}}, which is confirmed using Multisim as seen in \textit{Figure~\ref{fig:bode}}.

$$dB = 20log(100) = 40dB$$


\begin{figure}[h]
    \centering
    \includegraphics[width=1.0\textwidth]{graphics/bodeNew.png}
    \caption{The bode plot of the circuit.}
    \label{fig:bode}
\end{figure}

The bode plot of the circuit can be seen in \textit{Figure~\ref{fig:bode}}.
It shows the circuit has a dB gain of $\approx$ 40, for the entire bandwidth of 10 - 100kHz.



%Several simulations of the circuit were performed in Multisim.
%Which yielded the results seen in \textit{Figure~\ref{fig:InpVsOut10} -~\ref{fig:bode}}.
%Using the setup of the circuit seen in \textit{Figures~\ref{fig:Opamp1}~\ref{fig:Opamp2}~\ref{fig:Opamp3}~\ref{fig:Opamp4}}. 
%\textbf{LAGA MYNDIR EFTIR testinu}

%\begin{figure}[h]
%    \centering
%    \includegraphics[width=1.0\textwidth]{graphics/InpVsOut10hz.png}
%    \caption{With an input signal of 10mV RMS, at 10HZ.}
%    \label{fig:InpVsOut10}
%\end{figure}

%The simulation was run using an AC voltage generator which was set to output 10mV RMS and 10Hz.
%Which yielded the results seen in \textit{Figure~\ref{fig:InpVsOut10}}.
%\textbf{TALA UM MEIRA}

%\begin{figure}[h]
%    \centering
%    \includegraphics[width=1.0\textwidth]{graphics/InpVsOut100khz.png}
%    \caption{With an input signal of 10mV RMS, at 100kHZ.}
%    \label{fig:InpVsOut100k}
%\end{figure}

%\textbf{Bæta VIÐ 50khz multisim testi}

%The simulation was run using an AC voltage generator which was set to output 10mV RMS and 100kHz.
%Which yielded the results seen in \textit{Figure~\ref{fig:InpVsOut100k}}.
%\textbf{TALA UM MEIRA}

\clearpage

\subsection{Direct memory access}

When transferring data between main memories and I/O devices, direct memory access (DMA) can be utilized.
The benefit of using DMA for the data transfer is that it minimizes or eliminates the processor's involvement with the data transfer.
The processor only initializes the DMA controller by configuring the read and write memory, size of data for each transfer and I/O address.
In \textit{Figure~\ref{fig:DMAcontroller}} the process of the data transfer is better explained.

\begin{figure}[h]
    \centering
    \includegraphics[width=0.70\textwidth]{graphics/DMA.png}
    \caption{Data transfer for a DMA controller. Where the data transfers from the I/O device directly to memory. \cite{ahmed_design_2019}}
    \label{fig:DMAcontroller}
\end{figure}

A bus request (BR) signal is sent to the processor during the data transfer operation.
The processor then finishes its current job and replies with a bus grant (BG) signal.
The DMA controller receives the signal and can then initiate the data transfer.
There are two modes in which the DMA can perform data transfers.
The first is called flow-through, where the data flows through the DMA controller between the I/O device to memory.
The second is called fly by, where the data is transferred directly between the I/O device and memory \cite{ahmed_design_2019}.


%%% Local Variables: 
%%% mode: latex
%%% TeX-master: "DEGREE-NAME-YEAR"
%%% End: 
%%RUM: "Methods"
\part{The Second Part}
\chapter{Results}

%In this section you discuss any issues that came up while developing the system.  If you found something particularly interesting, difficult, or an important learning experience, put it here.  This is also a good place to put additional figures and data.


\textbf{Tala um raunverulega ásæðu af hverju ég vel þessa íhluti!!!!!!}

\section{Hydrophone}\label{sec:AquarionHydro}

\textbf{TALA UM AÐRA TÝPU SEM HEFÐI GETAÐ KOMIÐ TIL GREINA
https://www.nauta-rcs.it/English/Instruments/Hydrophones/CetaceanResearchTech/C55/C55.html}

First thing that needed to be determined was the hydrophone used.
Generally hydrophones are sensitive piezoelectric sensors that can detect small changes in pressure and convert that to electrical signal.
This project will use the Aquarian audio H1a for its low cost and good sensitivity as well as it was readily available at the Reykjavik University lab.

\begin{table}[h]\caption{Important specifications of the Aquarian H1a hydrophone.\cite{noauthor_aquarian_nodate}}.\label{Tab:Aquarian}
\begin{tabular}{l|l}
Sensitivity     & -190dB re: 1V/µPa (+/- 4dB 20Hz-4KHz) \\\hline
Useful range    & \begin{tabular}[c]{@{}l@{}}100KHz (not measured above 100KHz, approximate sensitivity\\  @100KHz = -220dB re: 1V/µPa)\end{tabular} \\\hline
Capacitance     & 25nF \\ \hline
Operating depth & \textless{}80meters\\ \hline
Cost            &  159\$ \\ \hline
\end{tabular}
\end{table}

There is no preamplifier or impedance buffer circuit within the Aquarian H1a hydrophone, so the circuit needs to amplify the output signal of the hydrophone.
The gain of the circuit depends on what the intended use of the hydrophone will be.
In this case for cetaceans and large aquatic wildlife, Aquarian recommends gain of 40 - 50 dB and for very distant sounds 60dB.

\begin{equation}
F_c = 1 / 0.000000157 * R  
\label{eq:fchydro}
\end{equation}

The Frequency response of the Aquarian H1a hydrophone can be calculated using \textit{Equation~\ref{eq:fchydro}} \cite{noauthor_aquarian_nodate}. 
Which is just \textit{Equation~\ref{eq:FC}} where the capacitance of the hydrophone has been added to it.


\subsection{Microcontroller og ADC}

\textbf{TALA UM TEENSY 3.5 og https://www.pjrc.com/teensy/K64P144M120SF5.pdf
}
\subsubsection{Microcontroller}
The Teensy3.5 was ultimately chosen for the microcontroller of the system.fa
This decision was made for several reasons.
The Teensy3.5 has two built-in ADC, which means the system might be able to add another hydrophone to it for recording.
Both ADCs have a maximum resolution of 16bits with a voltage range of 0 - 3.3V, or $\frac{3.3}{2^{16}} \approx 0.00005V \approx 50\mu V$ step size.
The Teensy also consumes little power, when running with no peripherals active the processor is powered roughly by 0.5mA or 0.165W.
Which is under the requirements set in the beginning of the project.
As well as it was readily available at the Reykjavik University lab.



\subsubsection{Operational amplifier}
\textbf{Mögulega tala um í kafla 2 hvað mikilvægustu þáttir opampa fyrir þetta verkefni var}

The operational amplifier that was chosen for this project was the OPA1644 from Texas Instruments.
It was chosen its advertised superior sound quality.
Where the noise is $\frac{5.1nV}{\sqrt{Hz}}$ at 1kHz.
%\begin{figure}[h]
%    \centering
%    \includegraphics[width=0.7\textwidth]{graphics/noiseDensvsFreq.png}
%    \caption{\textbf{BÆTA}}
%    \label{fig:noiseDensvsFreq}
%\end{figure}
%Which means 
%https://www.designnews.com/what-nvvhz-noise
%\textbf{KYNNA Sér SNR}
%$$\sqrt{100*10^3Hz - 10Hz} = 3\sqrt{11110}Hz$$
%$$3\sqrt{11110}Hz*5.1*10^{-9}V = 1.6*10^{-6}V$$
%$$ 1.6*10^{-6}V * 100 = 1.6*10^{-4}$$
%Using a 1V output (0dBV) from the op amp the signal to noise ratio is:
%$$SNR = 20log_{10}(\frac{1V}{1.6*10^{-4}V}) \approx 76dB$$
Low distortion of 0.00005\% at 1 kHz and a high slew rate of $\frac{20V}{\mu s}$.
As well as having a supply range that fits within the voltage range ($\pm2.25~-~\pm18$) of the Teensy analog pins of 3.3V\cite{noauthor_opa164x_nodate}.
It also has up to 40dB gain at 100kHz as seen in \textit{Figure~\ref{fig:dbvsFreq}}.

\begin{figure}[h]
    \centering
    \includegraphics[width=0.7\textwidth]{graphics/dbVsfreq.png}
    \caption{Gain and phase shift vs frequency\cite{noauthor_opa164x_nodate}}
    \label{fig:dbvsFreq}
\end{figure}



\section{Circuit}\label{sec:CircResult}

From the project goals the circuit needs to be able to gather data from signals with frequencies between 10 - 100kHz. 
The Aquarian H1a hydrophone data sheet also specifies that for cetacean vocalization recording the gain of the preamplifier needs to be between 40 - 50 dB.
The circuit will there for consist of active low pass filter and high pass filter, a scaling summing amplifier and an inverting op amp.
It will then feed to the Teensy 3.5 built-in ADC.
The setup of the circuit can be seen in \textit{Figures \ref{fig:Opamp1}, \ref{fig:Opamp4}}.
\textit{Figure \ref{fig:Opamp1}} shows the first part of the circuit.
The output of the hydrophone is first connected to a high pass filter, where the capacitance, 25nF of the hydrophone is used as the capacitor in the high pass filter.

\begin{figure}[h]
    \centering
    \includegraphics[width=0.70\textwidth]{graphics/OPamp1.png}
    \caption{The first part of the circuit where the output of the hydrophone first filtered using an active high pass filter and then connected to a operational amplifier and amplifying the signal by 20.}
    \label{fig:Opamp1}
\end{figure}


The desired value for the high pass filter was 10Hz and using \textit{Equation~\ref{eq:fchydro}} provided in the data sheet, the resistor value was estimated.   
$$F_c = \frac{1}{0.000000157 * 636k\Omega} = 10Hz$$
Using standard resistor values, the closest resistor value is 620k which yields a cut off frequency of 10.26Hz.
The gain over the active high pass filter can be represented using \textit{Equation \ref{eq:DCGain}} and \textit{Equation \ref{eq:ActiveHighPass}}. 
Where $A_{V1} = 1 + \frac{19k\Omega}{1k\Omega} = 20$.
$$A_f = \frac{(1+\frac{19k}{1k})(\frac{f}{10.26Hz})}{\sqrt{1 + (\frac{f}{10.26Hz})^2}} \approx 20$$
Which is approximately equal to 20, as seen in \textit{Figure~\ref{fig:AVhighpass}} over the entire bandwidth.

\begin{figure}[h]
    \centering
    \includegraphics[width=0.7\textwidth]{graphics/Av_Highpass.png}
    \caption{The gain of the operational amplifier with the active high pass filter.}
    \label{fig:AVhighpass}
\end{figure}

\textit{Figure \ref{fig:Opamp2}} shows the active low pass filter configuration which has a cut off frequency of $\approx$ 100kHz and a gain of 2.5.

\begin{figure}[h]
    \centering
    \includegraphics[width=0.7\textwidth]{graphics/OPamp2.png}
    \caption{The second operational amplifier, where the signal is first filtered by the active low pass filter as well as amplifying the signal.}
    \label{fig:Opamp2}
\end{figure}

The desired value for the high pass filter was 100kHz and using \textit{Equation~\ref{eq:FC}} and choosing a resistor value of 160$\Omega$, the capacitor value was estimated.   
$$F_c = \frac{1}{2\pi 9.9nF 160\Omega} = 100kHz$$
Using standard capacitor values, the closest capacitor value is 10nF which yields a cut off frequency of 99472Hz.
The gain over the active low pass filter can be represented using \textit{Equation \ref{eq:DCGain}} and \textit{Equation \ref{eq:ActiveLowPass}}. 
Where $A_{V1} = 1 + \frac{1.5k\Omega}{1k\Omega} = 2.5$.
$$A_f  = \frac{1 + \frac{1.5k\Omega}{1k\Omega}}{\sqrt{1 + (\frac{f}{99471.8Hz})^2}}$$
Which is approximately equal to 2.5 with frequencies until 20kHz, as seen in 
\textit{Figure~\ref{fig:AVlowpass}}.
\\
\begin{figure}[h]
    \centering
    \includegraphics[width=0.7\textwidth]{graphics/Av_Lowpass.png}
    \caption{The gain of the operational amplifier with the active low pass filter.}
    \label{fig:AVlowpass}
\end{figure}

\textit{Figure~\ref{fig:Opamp3}} shows the scaling summing operational amplifier which is used to shift the input signal by 1.65Vdc. 


\begin{figure}[h]
    \centering
    \includegraphics[width=0.70\textwidth]{graphics/OPamp3.png}
    \caption{The scaling summing operational amplifier, where the signal is shifted by 1.65Vdc}
    \label{fig:Opamp3}
\end{figure}


Which is important since the analog pins of the Teensy 3.5 have a voltage range of 0-3.3V.
There for the signal needs to be shifted by 3.3V/2 = 1.65Vdc.
%To estimate the voltage output of the op amplifier, \textit{Equation \ref{eq:invertDCGain}} and \textit{Equation \ref{eq:ScalingGain}} was used.
%For an input signal of 10mV amplitude, of the scaling summing op amp would be \textbf{KANSKI EKKI HAFA ?!?!?!} 
%$$V_{out} = -180\Omega(\frac{1.65V}{180\Omega} + \frac{\pm 10mV}{90\Omega}) = \pm $$
A voltage divider was needed to determine the resistor values for the 1.65Vdc which would ultimately put some design restraints on R16 and R13 seen in \textit{Figure~\ref{fig:Opamp3}}.
Firstly it was decided to have $R10 = 100\Omega$  and $R11 = 225\Omega$ to start off the calculations.
From that R14 could be calculated as $180\Omega$.
$$3.3V = \frac{\frac{1}{\frac{1}{180\Omega}+\frac{1}{225\Omega}}}{100\Omega+\frac{1}{\frac{1}{180\Omega}+\frac{1}{225\Omega}}} = 1.65V$$
Which was changed to $160\Omega$ because of the addition of R16 and using Multisim it was determined that $160\Omega$ would yield the closest results to the 1.65Vdc.
The gain for the input signal from the low pass filter is found by using \textit{Equation \ref{eq:invertDCGain}}
$$A_{V3} = -\frac{180\Omega}{90\Omega} = -2$$
The operational amplifier is in an inverting configuration which turns positive voltage signals to negative.
This needs to be rectified, since the Teensy 3.5 analog pins do not have a negative voltage range.
Which is done by another inverting operational amplifier, seen in \textit{Figure~\ref{fig:Opamp4}}.

\begin{figure}[h]
    \centering
    \includegraphics[width=0.70\textwidth]{graphics/OPamp4.png}
    \caption{The final operational amplifier, which is in an inverting configuration.}
    \label{fig:Opamp4}
\end{figure}

\fxfatal{Bæta við 330ohm viðnámi í Teensyinn}

The gain of inverting amplifier is found using \textit{Equation \ref{eq:invertDCGain}}.
$$ A_{V4} = -\frac{1k \Omega}{1k \Omega} = -1$$
The total circuit therefore consists of four operational amplifiers, high pass and low pass filter.
\textbf{MÖGULEGA BÆTA JÖFNU 'I KAFLA 2}
The total gain of the circuit for the input signal is found by.

$$A_{tot} = A_1A_1A_2A_3A_4 = 20*2.5*-2*-1 = 100$$

Finding the total dB gain of the circuit can be found by \textit{Equation \ref{eq:DBGAIN}}, which is confirmed using Multisim as seen in \textit{Figure~\ref{fig:bode}}.

$$dB = 20log(100) = 40dB$$


\begin{figure}[h]
    \centering
    \includegraphics[width=1.0\textwidth]{graphics/bodeNew.png}
    \caption{The bode plot of the circuit.}
    \label{fig:bode}
\end{figure}



Several simulations of the circuit were performed in Multisim.
Which yielded the results seen in \textit{Figure~\ref{fig:InpVsOut10} -~\ref{fig:bode}}.
Using the setup of the circuit seen in \textit{Figures~\ref{fig:Opamp1}~\ref{fig:Opamp2}~\ref{fig:Opamp3}~\ref{fig:Opamp4}}. 
\textbf{LAGA MYNDIR EFTIR testinu}

\begin{figure}[h]
    \centering
    \includegraphics[width=1.0\textwidth]{graphics/InpVsOut10hz.png}
    \caption{With an input signal of 10mV RMS, at 10HZ.}
    \label{fig:InpVsOut10}
\end{figure}

The simulation was run using an AC voltage generator which was set to output 10mV RMS and 10Hz.
Which yielded the results seen in \textit{Figure~\ref{fig:InpVsOut10}}.
\textbf{TALA UM MEIRA}

\begin{figure}[h]
    \centering
    \includegraphics[width=1.0\textwidth]{graphics/InpVsOut100khz.png}
    \caption{With an input signal of 10mV RMS, at 100kHZ.}
    \label{fig:InpVsOut100k}
\end{figure}

\textbf{Bæta VIÐ 50khz multisim testi}

The simulation was run using an AC voltage generator which was set to output 10mV RMS and 100kHz.
Which yielded the results seen in \textit{Figure~\ref{fig:InpVsOut100k}}.
\textbf{TALA UM MEIRA}




The bode plot of the circuit can be seen in \textit{Figure~\ref{fig:bode}}.
It shows  circuit has a dB gain of $\approx$ 40, for the entire bandwidth of 10 - 100kHz.



\clearpage
\subsection{Code}


\textbf{BREYTA NÖFNUM}


All codes were developed in Arduino IDE using a teensyduino extension.
Since this project is about the data acquisition of cetacean vocalization, all codes were developed to read analog signal and write the value to a file on a SD card.
The SD card used was a Samsung EVO Plus microSDHC 32 GB, capable of 95MB/s read speed and 20MB/s write speed.
Which should be sufficient since the data is 2 bytes per sample which means in theory the card should be able to handle a peak of 10Msps.



\textbf{SKOÐAFYRIR UTERIKNGA Á CONVERSION T'IMA!!!!!!!!!!!!!!!!}
% https://www.pjrc.com/teensy/K64P144M120SF5RM.pdf bls 859

\subsubsection{ContinousAnalogRead}

The basic function of the code is simple, it should read a 64KB buffer of  analog values from the circuit as well as save the microseconds of each read in a different buffer.
Once the buffers are full, the program begins to write all the buffer to the SD card.
To configure the ADC the, the ADC library by pedvide was used \cite{villanueva_pedvideadc_2021}.
The library handles the configuration of the built-in ADC and should make that process easier, however it makes it a little harder to see what it exactly configures the ADC to be.
To use the library, the code must first create an ADC object via 
$ADC *adc = new ADC();$
which is then used to define the attributes of the ADC, which can be seen in lines 69 - 83 in~ \textit{Listing~\ref{src:ContAnalogRead}}. 
The reference voltage is set as 3.3V, to have a voltage range of 0 - 3.3V.
The ADC averaging is set to 0, which after testing was the fastest configuration.
The conversion speed of the ADC was set to the fastest setting for 16bits conversion or HIGH\_SPEED\_16BITS, which sets the ADC clock to $<= 12 MHz$.
Then the sampling speed was set to the fastest setting of VERY\_HIGH\_SPEED, which adds +0 cycles to ADC clock (ADCK).
The library can as well configure so that once a conversion occurs an isr is triggered which is done by the enable interrupts function, and ties the conversion of adc0 to void adc0\_isr.
To set the ADC to do a continuous conversion, startContinuous() is used and can be configured to a specific pin on the Teensy.
Finally the results of the conversion are found using analogReadContinuous().

Once the ADC is configured, the device can start reading analog values.
It collects the analog readings to a buffer as well as the timestamp of each reading.
Once the buffers are full, the program writes to a SD card.
It writes both values as decimal values to a text file on the card.
The analog read and writes to SD card can be seen in \textit{Figure~\ref{fig:ContAnalSpeed}}

\begin{figure}[h]
    \centering
    \includegraphics[width=0.50\textwidth]{graphics/ContinousReadSpeed.png}
    \caption{The read and write speeds of ContinuousAnalogRead.ino}
    \label{fig:ContAnalSpeed}
\end{figure}


The data recorded by the device with the ContinuousAnalogRead.ino as the code. 
With an input signal of 10mVpp and 50kHz frequency. 
It appears to run too fast for the ADC, seeing as it has several conversions made at each voltage level as seen in \textit{Figure~\ref{fig:ContAnalREsults}}, every three data point a new conversion occurs.
Even though it should wait until the ADC is ready for a new conversion it does not appear to do so.


\begin{figure}[h]
    \centering
    \includegraphics[width=1.0\textwidth]{graphics/COntANalogResults.png}
    \caption{Results from Teensy using ContinuousAnalogRead.ino.}
    \label{fig:ContAnalREsults}
\end{figure}

Just before the test a single reading of the output signal was taken with an oscilloscope which can be seen in \textit{Figure~\ref{fig:ContAnalOscillascope}}.

\begin{figure}[h]
    \centering
    \includegraphics[width=0.70\textwidth]{graphics/ContAnalogReadOscillascope.PNG}
    \caption{Oscilloscope readings of the output signal}
    \label{fig:ContAnalOscillascope}
\end{figure}

%\begin{figure}[h]
%    \subfloat[\textbf{Sub figure caption}]{\includegraphics[width=0.50\textwidth,height=0.40\textwidth]{graphics/COntANalogResults.png}}
%    \label{fig:ContAnalSpeedREsults}
%    \hfill
%    \subfloat[\textbf{Sub figure caption }]{\includegraphics[width=0.40\textwidth,height=0.40\textwidth]{graphics/ContAnalogReadOscillascope.PNG}}
%    \label{fig:ContAnalOscillascope}
%    \caption{Results from the device compared to the oscilloscope readings}
%%\end{figure}


\subsubsection{PDBContinuousAnalogRead.ino}

\textbf{Bæta við meiri lýsingu á mannamáli og kanski pseudo kóða}

PDBContinuousAnalogRead.ino utilizes three specific builtin peripherals on the Teensy3.5 in order to read analog signals, which are an ADC, a programmable delay block (PDB) and a DMA channel.
The PDB is an accurate timer that is used to trigger the ADC to do a analog to digital conversion.
Once each ADC conversion is complete DMA receives a trigger, the ADC value is then transferred using DMA to memory.
When initializing the DMA it needs to know a few things about the buffer to which the ADC value is being transferred to such as the size of the buffer and its address. 
This is important because DMA counts how many transfers have been made, and at a set number of transfers it triggers an ISR.
Since in this program the DMA buffer is a 2 dimensional array (2 by 256 to have each buffer 512 bytes in size).
The ISR is set to trigger when each of the buffer is full, meaning when the first buffer is full the ISR is triggered and its contents are moved to another storage buffer.
While that transfer is happening the destination address for the buffer was changed to the second buffer and the data transfer is continually happening while the first buffer is still transferring its contents to the storage buffer.
This is crucial in order to get a non blocking code and not loose any data.
Therefor the speed of the program is dependent on how fast the Teensy can move data from the DMA buffer to the storage buffer.
The main program is then continually writing from the storage buffer to the SD card in 512 byte chunks.

\begin{figure}[h]
    \centering
    \includegraphics[width=0.70\textwidth]{graphics/flowChart.png}
    \caption{A flow chart giving a visual representation of the function of the program}
    \label{fig:CodeFlow}
\end{figure}



\subsection{Testing}



%Skoða aftur á scopei færa langt frá trigger punknt ~50 sveiflur og skoða breytileika þar.
%skða líka með FFT
%Skoða með mismunandi söfnunartíðnum og bera saman og skoða powerið

%skoða innmerki  10k með 100k söfnunartíðni
%                20k með 200k söfnunartíðni
%                30k með 300k söfnunartíðni

%Hvað ég safna rétt í tíma
%hve mikið hljóð
%setja inn merki, mæla útmerkið með FFT á scopei.
%skoða hljóð frá generatornum og svo frá merkinu með fft
%sjá mynd, ef það koma eh suð toppar á fft frá bara generator, sjá hvort þeir detta niður með RC filter ef ekki þá er suðið að koma %frá kerfinu.



%\textbf{looking at the oscilloscope on the builtin led, triggering it to switch states each time ISR of individual peripheral is %triggered. 
%Changing ISR only pdb = stable till sampleFreq  roughly 1MH  (491.7kHz)+- 0.3kHz measured by scope
%for 16 bits:
%ISR pdb adc = stable till sampleFreq roughly 280kHz   (139.5kHz) +- 0.01kHz measured by scope
%whole system = stable till roguhly 300khz samplefreq 
%Changing the resolution had minimal gains.
%It seemed to be able to be triggered a little bit faster however those were not stable for the teensy and the program would crash.}

Multiple tests were performed on the device, all equipment used for the tests can be seen in the list below.
\begin{itemize}
    \item \textit{Rhode \& Schwarz RTB20004} digital oscilloscope with a 2.5 Gsps sampling rate for wave form confirmation.
    \item \textit{Rigol DG1022} wave form generator capable of generating signals upto 25MHz sine wave and a resolution of 1$\mu Hz$ and down to 2mVpp.
    \item \textit{Rigol DP831} programmable DC power supply.
\end{itemize}



\subsubsection{Frequency of PDB, ADC and the DMA}

Several tests were made in order to find the maximum frequency at which parts of the device could remain relatively stable.
These tests would use an ISR, triggered by either the PDB or the ADC.
The ISR would either turn on the buildin LED on the Teensy or turn it off, depending on the whether the LED was on or off.
Which for every other trigger would make the LED blink.
Three test cases were examined, first using just the PDB which would trigger the ISR. 
Secondly the PDB and ADC together and finally for when the entire system was operational, for both cases the ISR would be triggered by the end of a ADC conversion.
The setup of the test can be seen in \textit{Figure~\ref{fig:SetupCircSpeed}}, the oscilloscope was connected to the builtin LED and the Teensy3.5 was powered via USB and the op amps were powered by the DC power supply.
The scope was set to take measurements of the time and frequency between the rising edge of rectangular signal that formed from blinking the LED. 
The oscilloscope shows the mean, maximum and minimum values for both, which are actually halved since the scope counts the rising edge of the pulses.
As well as the cursor was set over a single pulse to show the actual frequency of the ISR trigger speed.
The device was stopped when the oscilloscope had made roughly 10k wave count.
The setup was the same for all three test cases.

\fxfatal{Taka betri mynd}

\begin{figure}[h]
    \centering
    \includegraphics[width=0.7\textwidth]{graphics/SetupCircSpeed.jpg}
    \caption{The setup of the circuit for frequency for three test first just initializing the PDB, then the PDB and ADC and the third where the whole system was initialized.}
    \label{fig:SetupCircSpeed}
\end{figure}

In the first test, just the PDB timer was initialised and the ISR would trigger from the PDB timer.
Three set frequencies were tested, 1.2MHz, 600kHz and 300kHz.
The results of the tests can be seen in \textit{Figures~\ref{fig:PDBSp1200} - \ref{fig:PDBsp300}}.
The maximum frequency at which the PDB could run at was when the set frequency was set as 1.2MHz.
As explained before the actual frequency values from the oscilloscope are halved from the real values, so all measured frequency statistics are doubled, while the time of the period is halved.
The mean frequency was 1.17MHz, were the maximum was 1.27MHz and minimum of 767.7kHz with a standard deviation of 27.1kHz. 
The mean time for a period was 852ns with a standard deviation of 27ns as seen in \textit{Figure~\ref{fig:PDBSp1200}}.

\begin{figure}[h]
    \centering
    \includegraphics[width=0.8\textwidth]{graphics/STAT01_1200.PNG}
    \caption{The blinking of the LED, set frequency in the code was 1.2MHz}
    \label{fig:PDBSp1200}
\end{figure}

For the second test, the set frequency was 600kHz.
The mean frequency was 593.5kHz, were the maximum was 654.5kHz and minimum of 470.7kHz with a standard deviation of 8.42kHz. 
The mean time for a period was 1.68$\mu$s with a standard deviation of 29ns as seen in \textit{Figure~\ref{fig:PDBSp600}}.

\begin{figure}[h]
    \centering
    \includegraphics[width=0.8\textwidth]{graphics/STAT02_600.PNG}
    \caption{The blinking of the LED, set frequency in the code was 600kHz}
    \label{fig:PDBSp600}
\end{figure}

For the third test, the set frequency was 300kHz.
The mean frequency was 298.4kHz, were the maximum was 320.7kHz and minimum of 122.4kHz with a standard deviation of 1.63kHz. 
The mean time for a period was 3.35$\mu$s with a standard deviation of 21ns as seen in \textit{Figure~\ref{fig:PDBsp300}}.

\begin{figure}[h]
    \centering
    \includegraphics[width=0.8\textwidth]{graphics/STAT03_300.PNG}
    \caption{The blinking of the LED, set frequency in the code was 300kHz}
    \label{fig:PDBsp300}
\end{figure}

The next test was for the ADC, but now by initializing both the ADC and the PDB.
The ISR would also be set to trigger when completing an ADC conversion.
The set frequency was 280kHz, because at 300kHz the program would stop running after roughly 2 seconds.
The mean frequency was 279.1kHz, were the maximum was 208.3kHz and  minimum of 225.5kHz with a standard deviation of 1.1kHz. 
The mean time for a period was 3.58$\mu$s with a standard deviation of 14ns as seen in \textit{Figure~\ref{fig:PDBADCDMAsp280}}.

\begin{figure}[h]
    \centering
    \includegraphics[width=0.8\textwidth]{graphics/STATADC_280.PNG}
    \caption{The blinking of the LED when completing an ADC conversion triggered the ISR, set frequency in the code was 280kHz}
    \label{fig:PDBADCDMAsp280}
\end{figure}

The next test was for the whole system was initialized.
The ISR is still set to trigger when completing an ADC conversion.
The mean frequency was 298.3kHz, were the maximum was 532.4kHz and minimum of 208.5kHz with a standard deviation of 5.4kHz. 
The mean time for a period was 3.39$\mu$s with a standard deviation of 69.6ns as seen in \textit{Figure~\ref{fig:PDBADCDMAsp280}}.

\begin{figure}[h]
    \centering
    \includegraphics[width=0.8\textwidth]{graphics/ALLT300k.PNG}
    \caption{The blinking of the LED when completing an ADC conversion triggered the ISR, set frequency in the code was 300kHz}
    \label{fig:PDBADCDMAsp300}
\end{figure}


\clearpage




\subsubsection{Noise testing}

%setting https://picture.iczhiku.com/resource/eetop/SHkHEFiIIDHlSccB.pdf

Several tests were made in order to estimate the noise of the device.
Fast Fourier transform (FFT) function of the RTB2004 oscilloscope was used for the measurements and the DC power supply was used to power the op amps.
The oscilloscope was set to AC coupled for the best broadband range measurement%https://training.ti.com/ti-precision-labs-op-amps-noise-measuring-system-noise 
, attenuator set to 1:1 ratio %https://www.testandmeasurementtips.com/reduce-oscilloscope-noise-measurements/
for accurate noise measurements and the Hannig window was used for 
Two test cases where examined, first when a sine wave was the input of the circuit and secondly when the input was connected to ground.
For both cases the probe was connected to the output of the op amps.
The setup for the tests can be seen in \textit{Figure~\ref{fig:SetupFFT}}.

\textbf{%https://web.sonoma.edu/esee/courses/ee442/archives/sp2019/supp/defining_dBu.pdf
- skoða} 
%https://www.youtube.com/watch?v=oLBGNC9FGwo&ab_channel=TexasInstruments minuta 3:47

\begin{figure}[h]
    \centering
    \includegraphics[width=0.7\textwidth]{graphics/TESTINGwSineinp.jpg}
    \caption{Configuration of the noise tests where the circuit is tested, where the probe is connected to the output of the op amps and the input signal is either a sine wave or its connected to ground.}
    \label{fig:SetupFFT}
\end{figure}







The bandwidth of the FFT was set at 100kHz with an input signal being a sine wave with 10$mV_{pp}$ and 20kHz frequency being sent to the input of the op amps.
The results can be seen in \textit{Figure~\ref{fig:Noise20k10mVpp100kband}}, at 20kHz the input signal is apparent with a peak of 3dBm while the noise floor is around -72dBm.
Which yields a SNR, which is the difference between the two amplitudes as 75dB.

\clearpage

\begin{figure}[h]
    \centering
    \includegraphics[width=0.7\textwidth]{graphics/Noise20k10mVpp100kband.PNG}
    \caption{FFT on the RTB2004 oscilloscope. An input sine wave signal of 10$mV_{pp}$ and a frequency of 20kHz seen at the top of the figure and the respective FFT below it with a 100kHz bandwidth.}
    \label{fig:Noise20k10mVpp100kband}
\end{figure}



%\begin{figure}[h]
%    \centering
%    \includegraphics[width=0.7\textwidth]{graphics/Noise20kinp300kBand.PNG}
%    \caption{FFT on the RTB2004 oscilloscope. An input sine wave signal of 10$mV_{pp}$ and a frequency of 20kHz seen at the top of the figure and the respective FFT below it with a 300kHz bandwidth.}
%    \label{fig:Noise20k10mVpp300kband}
%\end{figure}


The same test was performed for just the generator, bypassing the op amp circuit and connecting it directly to the oscilloscope.
The results can be seen in \textit{Figure~\ref{fig:NoiseGenerator20kInp100kBand}}, at 20kHz the input signal is apparent with a peak of -36.8dBm while the noise floor is around -116dBm.
Which yields a SNR of 80dB.

\begin{figure}[h]
    \centering
    \includegraphics[width=0.7\textwidth]{graphics/NoiseGenerator20kInp100kBand.PNG}
    \caption{FFT on the RTB2004 oscilloscope connected directly to the signal generator. 
    An input sine wave signal of 10$mV_{pp}$ and a frequency of 20kHz seen at the top of the figure and the respective FFT below it with a 100kHz bandwidth.}
    \label{fig:NoiseGenerator20kInp100kBand}
\end{figure}

\clearpage

%\begin{figure}[h]
%    \centering
%    \includegraphics[width=0.7\textwidth]{graphics/NoiseGenerator20kInp300kBand.PNG}
%    \caption{FFT on the RTB2004 oscilloscope. 
%    An input sine wave signal of 10$mV_{pp}$ and a frequency of 20kHz seen at the top of the figure and the respective FFT below it with a 300kHz bandwidth.}
%    \label{fig:NoiseGenerator20kInp300kBand}
%\end{figure}


Now for when the input signal was connected to ground.
The op amps are still getting power from the DC power supply and the probe is connected to the output of the op amps.
The results can be seen in \textit{Figure~\ref{fig:noisefloor100k}}.
This measurement yields a noise floor around of 87.1dBm.


\begin{figure}[h]
    \centering
    \includegraphics[width=0.7\textwidth]{graphics/NoiseFloor100k.PNG}
    \caption{General noise created by the circuit when op amps are powered and the input is grounded over a frequency band of 100kHz}
    \label{fig:noisefloor100k}
\end{figure}

%Now for when the bandwidth is set to 300kHz.
%Which can be seen in \textit{Figure~\ref{fig:noisefloor100k}}.
%This measurement yields a noise floor of around 86.8dBm.

%\begin{figure}[h]
%    \centering
%    \includegraphics[width=0.7\textwidth]{graphics/NoiseFloor300k.PNG}
%    \caption{General noise created by the circuit when op amps are powered and the input is grounded over a frequency band of 300kHz}
%    \label{fig:noisefloor300k}
%\end{figure}




\clearpage





\subsubsection{Recording signals from a generator}

All tests had the same setup of the device.
The Teensy3.5 was powered via USB, the probe was connected to where input signal was being read and ground was connected to analog reference of the Teensy3.5.
The op amps where powered by the as before by a DC power supply and the input signal was generated by the wave form generator.
The setup of the device with everything connected can be seen in \textit{Figure~\ref{fig:testCircSetup}}.
The Teensy would recorded the record the signal to the SD card, where the values were in hexadecimal.
The output signal of the op amps was then recorded by the RTB2004 oscilloscope.
Ones it had stopped recording the data was transformed from hexadecimal to decimal using the python script seen in \textit{Listing~\ref{src:converHex}} and then plotted using Matlab. script seen in \fxfatal{Bæta Matlab kóða Kanski ekki ???}.


\begin{figure}[h]
    \centering
    \includegraphics[width=0.7\textwidth]{graphics/TestSetup.jpg}
    \caption{The setup for the circuit for the tests which used the wave form generator as the input signal.}
    \label{fig:testCircSetup}
\end{figure}

In this test it is simulating a 170dB as a source level.
Using the Aquarian H1a as the hydrophone the voltage output would equate to 10mV $V = 10^{170dB-190dB/10} = 10mV$, which was used for all the following tests.
The difference between the three tests was the input signals frequency and the sample frequency of the Teensy.
The first test had the Teensys sample frequency set at 100kHz and a input signal of 10kHz was used.
The data recorded by the Teensy can be seen in \textit{Figure~\ref{fig:Teensy10k100k}}.
While \textit{Figure~\ref{fig:Oscillo10k100k}} shows respective results from the RTB2004 oscilloscope.

\begin{figure}[h]
    \centering
    \includegraphics[width=1.0\textwidth]{graphics/10kin_100ksampl.png}
    \caption{Test of the circuit where the input signal was a sinusoidal wave with a frequency of 10kHz and an amplitude of $10mV_{pp}$ and the sample frequency was set as 100kHz}
    \label{fig:Teensy10k100k}
\end{figure}

\begin{figure}[h]
    \centering
    \includegraphics[width=0.7\textwidth]{graphics/10k10mvPP100ksamp.PNG}
    \caption{Shows the results of the oscilloscope scoping, where the probe was connected to the output of the op amp and the input signal was 10kHz.}
    \label{fig:Oscillo10k100k}
\end{figure}

\clearpage



Next the sample frequency was increased to 200kHz and the input signal was 20kHz.
The results of the data recorded by the Teensy can be seen in\textit{Figure~\ref{fig:Teensy20k200k}}.
\textit{Figure~\ref{fig:Oscillo20k200k}} shows respective results from using the oscilloscope for the same input signal as before.

\begin{figure}[h]
    \centering
    \includegraphics[width=1.0\textwidth]{graphics/20kin_200ksampl.png}
    \caption{Test of the circuit where the input signal was a sinusoidal wave with a frequency of 20kHz and an amplitude of $10mV_{pp}$ and the sample frequency was set as 200kHz}
    \label{fig:Teensy20k200k}
\end{figure}

\begin{figure}[h]
    \centering
    \includegraphics[width=0.7\textwidth]{graphics/20k10mvPP200ksamp.PNG}
    \caption{Shows the results of the oscilloscope scoping, where the probe was connected to the output of the op amp and the input signal was 10kHz}
    \label{fig:Oscillo20k200k}
\end{figure}

\vspace{4cm}



In \textit{Figure~\ref{fig:Teensy30k300k}} the data that the Teensy recorded at 300kHz sample frequency, where the input signal was 30kHz.
\textit{Figure~\ref{fig:Oscillo30k300k}} shows respective results from using the oscilloscope for the same input signal as before.

\begin{figure}[h]
    \centering
    \includegraphics[width=1.0\textwidth]{graphics/30kin_300ksampl.png}
    \caption{Test of the circuit where the input signal was a sinusoidal wave with a frequency of 30kHz and an amplitude of $10mV_{pp}$ and the sample frequency was set as 300kHz}
    \label{fig:Teensy30k300k}
\end{figure}

\begin{figure}[h]
    \centering
    \includegraphics[width=0.7\textwidth]{graphics/30k10mvPP300ksamp.PNG}
    \caption{Shows the results of the oscilloscope scoping, where the probe was connected to the output of the op amp and the input signal was 30kHz}
    \label{fig:Oscillo30k300k}
\end{figure}

\clearpage



Two more tests were performed to better estimate the Teensys accuracy, since at 10mV the signal generator was quite inconsistent and created some variance in its output. \fxfatal{ert með mynd af mælingunni kanski að bæta??}
The test setup can be seen in \textit{Figure~\ref{fig:Last2TestsSetup}}, where the input signal bypasses the op amp circuit and goes straight to the Teensy and Oscilloscope.

\begin{figure}[h]
    \centering
    \includegraphics[width=0.7\textwidth]{graphics/Last2Tests.jpg}
    \caption{The input signal connected to the 330$\Omega$ resistor straight to the Teensy.}
    \label{fig:Last2TestsSetup}
\end{figure}



\vspace{4cm}


The first signal was a sine wave with 1$V_{pp}$ and VDC offset of 1.65V and 50kHz frequency.
The data plotted from the Teensy can be seen in \textit{Figure~\ref{fig:OscilloCompTeensyAC}}.
Where the average from the Teensy data has an average voltage of 1.6462V, while the oscilloscope shows 1.6504V.

\begin{figure}[h]
    \centering
    \includegraphics[width=1.0\textwidth]{graphics/OscilloTeensyAC50k1vpp165voffpng.png}
    \caption{With a AC signal of 1$V_{pp}$ and 1.65V DC offset, Teensys readings at 300ksps compared to the RTB2004 oscilloscope at 2.5Gsps}
    \label{fig:OscilloCompTeensyAC}
\end{figure}

The second input signal was DC 1.65V.
The data plotted from the Teensy can be seen in \textit{Figure~\ref{fig:OscilloCompTeensyDC}}.
Where the average from the Teensy data has an average voltage of 1.6727V, while the oscilloscope shows 1.6695V.


\begin{figure}[h]
    \centering
    \includegraphics[width=1.0\textwidth]{graphics/OscilloTeensyDC165Read.png}
    \caption{With a DC signal of 1.65V, Teensys readings at 300ksps compared to the RTB2004 oscilloscope at 2.5Gsps}
    \label{fig:OscilloCompTeensyDC}
\end{figure}

%https://www.eevblog.com/forum/beginners/help-with-rapid-adc-data-aquizition/25/
%https://forum.pjrc.com/threads/30171-Reconfigure-ADC-via-a-DMA-transfer-to-allow-multiple-Channel-Acquisition?p=140300#post140300

%TESTa adc_dma_timer í arduino example segja hve hátt þett getur samplað.

%https://github.com/pedvide/ADC/blob/master/AnalogBufferDMA.cpp -> prufa þetta

%%% Local Variables: 
%%% mode: latex
%%% TeX-master: "DEGREE-NAME-YEAR"
%%% End: 
%%RUM: "Results"
\chapter{Discussion}


\section{Summary}


\section{Conclusion\label{sec:conclusions}}

%%% Local Variables: 
%%% mode: latex
%%% TeX-master: "DEGREE-NAME-YEAR"
%%% End: 
%%RUM: "Discussion"






\fxfatal{LAGA REFERENCEA}
%% ---------------------------------------------------------------
\printbibliography{} %%RUM: "References"

%% If appendices are needed, uncomment the following line
%% and include the appendices in separate files
\appendix{}%%RUM: "Appendicies (as appropriate)

\chapter{Cetacean vocalization}\label{appTab:cetaceanVocalization}
\begin{table}[]\caption{Frequency ranges of whales that are found near Iceland \cite{richardson_marine_2013}, * taken from \cite{richardson_effects_1991}}.\label{Tab:WhaleHz}
\centering
\resizebox{\textwidth}{!}{
\begin{tabular}{|l|l|l|l|l|}
\hline
\begin{tabular}[c]{@{}l@{}}Species \\ of \\ whale\end{tabular} & Noise type             & \begin{tabular}[c]{@{}l@{}}Frequency range\\ {[}Hz{]}\end{tabular} & \begin{tabular}[c]{@{}l@{}}Dominant \\ frequencies \\ {[}Hz{]}\end{tabular} & \begin{tabular}[c]{@{}l@{}}Source level \\ {[}dB re 1 $\mu$Pa at 1m{]}\end{tabular} \\ \hline
Blue    & Moans & 12 - 390 & 16 - 26 & 188 \\ \cline{2-5}
        & Clicks & \begin{tabular}[c]{@{}l@{}}  6k - 8k\\ 21k - 31k\end{tabular} & \begin{tabular}[c]{@{}l@{}}6k - 8k\\ 25k\end{tabular} & \begin{tabular}[c]{@{}l@{}}130\\ 159\end{tabular} \\ \hline
\begin{tabular}[c]{@{}l@{}}Bottlenose \\dolphin\end{tabular}    & Whistles & 0.8 - 24 & 4.5 - 14.5 & 125 - 173 \\ \cline{2-5}
        & \begin{tabular}[c]{@{}l@{}}Low-freq. \\narrowband\end{tabular}    & < 2 & 0.3 - 0.9 & - \\ \hline
Fin     & Moans, down sweeps & 14 - 118 & 20 & 160 - 186 \\ \cline{2-5}
        & Constant call & 20 - 40 & - & - \\ \cline{2-5}
        & Moans, tones, up sweeps & 30 - 750 & - & 155 - 165 \\ \cline{2-5}
        & Rumble & 10 - 30 & - & - \\ \cline{2-5}
        & Whistles, chirps & 1.5k - 5k & 1.5k - 2.5k & - \\ \cline{2-5}
        & Clicks & 16k - 28k & - & - \\ \hline
Harbour porpoises & Clicks & 2 & - & 100 \\ \cline{2-5}
        & Echolocation & 110k - 150k & - & 135 - 177 \\ \hline
Humpback & Song components & 30 - 8k & 120 - 4k & 144 - 174 \\ \cline{2-5}
        & Shrieks & - & 750 - 1.8k & 179 - 181 \\\cline{2-5}
        & Horn blasts & - & 410 - 420 & 181 - 185 \\ \cline{2-5}
        & Moans & 20 - 1.8k & 35 - 360 & 175 \\\cline{2-5}
        & Grunts & 50 - 1.9k+ & - & 190 \\\cline{2-5}
        & Pules trains & 25 - 1.25k & 25 - 80 & 179-181 \\ \cline{2-5}
        & Underwater blows & 100 - 2k & - & 158 \\\cline{2-5}
        & Fluke and flipper slap & 30 - 1.2k & - & 183-192 \\\cline{2-5}
        & Clicks & 2k - 8.2k & - & - \\ \hline
Minke & Down sweeps & 60 - 130 & - & 165 \\ \cline{2-5}
        & Moans, grunts & 60 - 140 & 60 - 140 & 151 - 175 \\\cline{2-5}
        & Ratchet & 850 - 6k & 850 & - \\ \cline{2-5}
        & Clicks & 3.3k - 20k & \textless{}12k & 151 \\ \cline{2-5}
        & Thump trains & 100 - 2k & 100 - 200 & - \\ \hline
Killer  & Whistles & 1.5k - 18k & 6k - 12k & - \\ \cline{2-5}
        & Pulsed calls & 500 - 25k & 1k - 6k & 160 \\ \cline{2-5}
        & Echolocation &  30 - 100kHz * &12k - 25k & 180 \\ \hline
Sei     & FM sweeps & 1.5k - 3.5k & - & - \\ \hline
Sperm    & Clicks & 0.1 - 30 & 2 - 4, 10 - 16 & 160 - 180 \\ \hline
\begin{tabular}[c]{@{}l@{}}White-beaked\\dolphin\end{tabular} & Squeals & - & 8 - 12 & - \\ \hline

\end{tabular}}
\end{table}

\chapter{Setting of the registers}\label{sec:codeExplain}
\fxfatal{Eitthvað að kikja á }
To configure Teensy's ADC, DMA, PDB, the datasheet of the Kinetis K64F reference manual was used \cite{freescale_semiconductor_kinetis_2021}.
The information regarding the registers of each module can be found among other things.

%\textbf{BLS 933 \cite{freescale_semiconductor_kinetis_2021}.}
Starting with the PDB  which can be seen in \textit{Listing \textbf{BÆTA VIÐ kóða}}.
Firstly the clock for the PDB clock needs to be enabled, which is done by setting the SIM\_SCGC6\_PDB;
Which for the Teensy3.5 is 60MHz default and can be scaled down using prescaler.
Since the desire was to trigger at a high frequency the prescaler and multiplier of the prescaler were both set as 1 for the high clock speed possible.
\textbf{$$input clock = \frac{F\_BUS}{\frac{prescaler}{/multiplier}} = 60MHz$$}
Then the modulus register needs to be set, which will define the frequency of which the PDB will trigger and was calculated using the equation below.

$$modulus = \frac{input clock}{sample frequency}$$

The input clock increases a counter at the rate of its speed and the modulus sets the value of which the counter will set at zero and restart its count if the PDB is in continuous mode.
Once the modulus has been determined, several registers need to be configured in order to define the PBDs operation.
Most of which is done via PDB0\_SC register, which is the status and control register for PDB on channel 0.
Firstly to enable the PDB $ PDB\_SC\_PDBEN $ is set.
Then the trigger input source is selected as software trigger, by selecting $0xf (15)$ in $ PDB\_SC\_TRGSEL $.
To have to PDB run until it is stopped, the continuous mode needs to be enabled which is done by enabling $ PDB\_SC\_CONT $.
The PDB can trigger an interrupt to do that PDB interrupts need to be enabled, which is done by enabling
$ PDB\_SC\_PDBIE $.
There needs to be set an interrupt delay when using the PDB for the interrupt, which is done by setting a value to $PDB0\_IDLY = 1;$ and the ISR is triggered when the PDB counter reaches idly value.
Once the PDB configuration has been set, the registers need to be updated which is done by $ PDB\_SC\_LDOK$.
Then a PDB pre-trigger needs to be enabled since it is used to precondition the ADC block prior to the actual ADC conversion occurs.
%\textbf{p942}
This is done by enabling the pre-trigger by enabling the $PDB0\_CH0C1\_EN$ as well as the channel pre-trigger output $PDB0\_CH0C1\_TOS$.
This is achieved by writing $0x101$ to PDB0\_CH0C1.
%\begin{figure}[h]
%    \centering
%    \includegraphics[width=0.70\textwidth]{graphics/PDBtrigger.png}
%    \caption{PDB ADC triggers and DAC interval triggers use case \textbf{KANSKI sleppa nennei ekki að tala um}\cite{freescale_semiconductor_kinetis_2021}}
%    \label{fig:PDBTrigger}
%\end{figure}
Then the ADC can be configured first, the conversion mode needs to be determined.
The built-in ADC is up to 16bit and can be configured to run at 8-, 10-, 12- and 16-bit resolution modes, which is done through $ADC\_CFG1\_MODE$.
Two different reference voltages can be set, for the configuration of the circuit analog ground pin is used instead of an internal ground.
To achieve this $ADC0\_SC2$ needs to be set to 0.
Since the ADC is supposed to PDB trigger, that needs to be enabled.
$ADC\_SC2\_ADTRG$ sets the ADC to be hardware triggered, which in this case is set to be the PDB.
DMA transfer needs to be enabled here so that the conversion values of the ADC can go straight to memory, which is done by enabling  $ADC\_SC2\_DMAEN$.
All that is left is to select the analog pin, which is done by writing the pin hex value to $ADC0\_SC1A$ which in this case was "A9" or 0x04, see page 120 in \cite{freescale_semiconductor_kinetis_2021}.
The results of each conversion are then stored in $ADC0\_RA$ data result register.

Lastly the DMA is configured, to help with the process "DMAChannel.h" was used, which is a Teensy library.
Multiple registers need to be configured and set.
For instance where the data is coming from, where it is supposed to be delivered and when to go into the interrupt service routine (ISR) to name a few.
To initiate a DMA channel DMA.begin() is called.
Then the trigger for when the DMA transfer is supposed to occur needs to be set, which is done by setting the service request trigger as when an ADC conversion occurs.
This is done by configuring the $dma.triggerAtHardwareEvent$ as seen here $dma.triggerAtHardwareEvent(DMAMUX\_SOURCE\_ADC0)$.
Which sets in motion the DMA transfer.
Defining the source address $dma.TCD->SADDR$ needs to be set in this case, the location of the results of the ADC conversion, $ADC0\_RA$.
If the source address changes between each DMA transfer, there needs to be an offset put on the address which is done via $dma.TCD->SOFF$. 
Which in this case is 0 bytes, since $ADC0\_RA$ is just overwritten for each ADC conversion and its address does not change.
If the source address changes between each ISR call then adjustments can be made through $dma.TCD->SLAST$, which also does not change and is still the address of $ADC0\_RA$.
The data that is incoming and supposed to be transmitted needs to be set, which in this project is 2 bytes since the ADC will be configured in an above 8bit resolution.
This is done by $dma.TCD->ATTR = DMA\_TCD\_ATTR\_SSIZE(1) | DMA\_TCD\_ATTR\_DSIZE(1);$ which defines 16bit data transfers.
Which firstly defines the size of the incoming data as well as defining the destination data transfer size.
The number of bytes to transfer for each service request $DMAMUX\_SOURCE\_ADC0$ which is done with  $dma.TCD->NBYTES_MLNO = 2;$, this is considered a minor loop .
Then the destination source needs to be set like before, $dma.TCD->DADDR = dmaBuf;$ defines the destination address to a DMA buffer called dmaBuf which is defined in the main program.
This time there needs to be an offset is set to the address since the DMA is writing to a buffer to store the value instead of just overwriting the value like it was for the source.
This is done by setting the destination offset as the size of the ADC conversion value, or 2 bytes $dma.TCD->DOFF = 2;$.
To keep track of how many values have been set through DMA, the  $dma.TCD->CITER\_ELINKNO $ counter is set as the size of the buffer that the DMA values are being transferred to.
This value is decremented each time a value has been transferred via service request and transfer.
Since in the main program dmaBuf is 2 arrays of a size of 128 indexes this is set to 128 \textbf{Staðfesta!}.
When the count reaches 0, meaning the DMA buffer is full.
The counter needs to be reset back to its original value, this is done by setting $dma.TCD->BITER\_ELINKNO$ as the same value as the counter was set as.
The destination of the last address adjustment for the next transfer needs to be set, which is done by $dma.TCD->DLASTSGA = -sizeof(dmaBuf);$
When both DMA buffers are full, DMA resets, so it can rewrite to the first buffer.
To define when the ISR occurs, the $dma.TCD->CSR $ is set.
Since the BITER register was also set the reference manual states to use $DMA\_TCD\_CSR\_INTMAJOR$, however this resulted in only half the data being transmitted so $DMA\_TCD\_CSR\_INTMINOR$ was also set.
This sets DMA to trigger twice once when the first DMA buffer is full as well as when the second buffer is full.

Finally the DMA channel is enabled by $dma.enable();$.
The DMA ISR is used to transfer the data from the dmaBuf to fifoBuf in order to transfer it to the SD card later on. 

\input{texts/code} % as an example, perhaps some of your code

%\backmatter{} % Sections after this don't get numbers
%% We prefer that all elements be numbered

%%%%%%%%%%%%% SHOW INDEX %%%%%%%%%%%%%%%%%%
%% Index, optional.  A good idea on longer documents

% You can put instructions at the beginning of the index:
%\renewcommand{\preindexhook}{%
%  The first page number is usually, but not always,
%  the primary reference to the indexed topic.\vskip\onelineskip}

%% You may have to run "makeindex <FILENAME>" to have it be generated
%% Depending upon which package you chose.
%% 
\clearforchapter{}
\printindex{}%%RUM: Not mentioned

%\backcover{}%%RUM: "Back cover (only Phd)
\end{document}

%% ---------------------------------------------------------------

%%% Local Variables:
%%% mode: latex
%%% TeX-master: t
%%% TeX-engine: xetex
%%% End:
