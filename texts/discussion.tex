\chapter{Discussion}

\section{Summary}

This thesis set out to look into the possibility of using a small, energy-efficient microcontroller to record cetacean vocalizations.
Different microcontrollers were explored and it was decided that the Tennsy was the best suited for the task.
Since it had a built-in ADC with 16bit resolution as well as an ADC library and even though not considered initially having DMA channels proved crucial to create the system.
An attempt was made to use the library however, that yielded poor results as seen in \textit{Figure~\ref{fig:ContAnalREsults}}. 
It would record a conversion even though the ADC was not ready for a new conversion.
So a decision was made to directly set the registers.
The final system operates by having a PDB timer trigger the ADC at precise intervals, and a DMA channel takes the data to memory where it can later be sent to the SD card.

Aquarian H1A hydrophone sets some recommendations for the preamplifier circuit. 
Such as the gain for cetacean monitoring (40dB gain ) as well as the maximum frequency of 100kHz, which can be seen in \textit{Section~\ref{sec:AquarionHydro}}.
The preamplifier was designed with that in mind and in \textit{Section~\ref{sec:CircResult}}, there it is shown that for both the gain required, as well as filtering signals that are not within the bandwidth of the hydrophones sensing abilities.

Several tests were made to assess the capabilities of the final device.
The first tests were regarding the capabilities of the PDB timer and its capability to trigger at a set frequency.
It is able to be triggered as fast as 1.2MHz.
However that is out of the ADC specs.
The system is able to run at 300kHz sampling frequency, with a sampling jitter of 69.6ns.
Which can result in a voltage standard deviation of 11.484mV.

%in order to maintain a 0.5LSB standard deviation.
%http://www.audiophilleo.com/zh_hk/docs/Dunn-AP-tn23.pdf - sampling jitter

Several noise tests were performed on both the entire device as well as the signal generator.
There was no special shielding used and the ground for the probe could act as an antenna these factors could all attribute to some extrinsic noise generation.
How ever in \textit{Figure~\ref{fig:Noise20k10mVpp100kband}} shows that an input signal of nominal value is not lost in noise, with a 75dB SNR which is more than sufficient as \textit{section~\ref{sec:DigitalAudiRec}} demonstrates.

Three tests were performed on the whole device where the input signal to the op-amps was a sine wave with a different frequency and for each test the sampling frequency was changed to be 10x that of the input signal.
This was done to see if the device was actually capable of 300kHz sampling.
In \textit{Figures~\ref{fig:Teensy10k100k},\ref{fig:Teensy20k200k} and~\ref{fig:Teensy30k300k}} the recorded results can be seen.
The device appears to be capable of sampling up to 300kHz, where it follows each signal nicely and has 10 data points for each period.

Two more similar tests were performed, since at 10$mV_{pp}$ the signal had a lot of variation compared to the actual signal, which could cause some errors.
So two more tests were performed, where the input signal was connected directly to the Teensy through a resistor.
Two signals were tested, first with a sine wave with a voltage offset of 1.65V, 1$V_{pp}$ so that the Teensy would not receive negative voltages and a frequency of 50kHz.
The comparison between the results from the Teensy compared to what the oscilloscope received can be seen in \textit{Figure~\ref{fig:OscilloCompTeensyAC}}.
The same test was performed with a DC 1.65V, the results can be seen in \textit{Figure~\ref{fig:OscilloCompTeensyAC}}. 
The results are quite good, the Teensy is clearly able to keep up with the signal and give accurate results.
There is a difference between the mean, maximum and minimum voltages. 
But that could be the results of not being exactly taken at the same time as well as the oscilloscope samples at a much higher rate which could influence the outcome.


\clearpage


\section{Conclusion}\label{sec:conclusions}

Currently the device is still in prototype stages where it is still on a breadboard and has yet to be proven with a hydrophone.

The preamplifier circuit is configured to have the Aquarian H1a hydrophone as its source for signal sensing, which can easily be replaced with a different one.
The bandwidth of signals can also be altered easily by changing the high- and low pass filter setup.
Since dominant frequencies of cetacean vocalizations of whales found near Iceland seem to be under 30kHz.
It also consumes very little power, where the op-amps in total need around 0.165W and the Teensy draws 0.25W so in total the total power consumption should be around 0.415W which is way under the requirements.
So in total the device consumes roughly 0.415W at 3.3V, which is way under the requirements set at the start.
The current configuration of the project costs around 190\$ with the hydrophone, op-amps and the Teensy3.5.
It is hard to find a price for a complete system capable of recording cetacean vocalization but considering that devices such as the RUDAR uses the CRT C57 hydrophone which costs around 1290€.
It is safe to say the total cost of the recording device is low cost relative to other products commercially available.

The device is currently capable of recording continuously at 300ksps for up to 16bit resolution, which in theory should be more than sufficient to record signals up to 150kHz according to the Nyquist-Shannon theorem.

It was however decided early on to see the maximum potential of the Teensy platform audio recording and see its capabilities, which can then beneficial to the researchers.
Because some research suggested vocalization occurring at higher frequencies as well as the echolocation noise can go well above 100kHz so the device could record at least some parts of those even though majority of the vocalization occurs at less than 30kHz.

The main concern is the sampling standard deviation of 69.6ns.
This is insufficient for 16bits resolution in the worst-case scenario,
where the input signal is 100kHz and has a $V_{pp}$ of 3.3V and when the ADC is taking a measurement marked by the black asterisk seen in \textit{Figure~\ref{fig:sine100k}} which has the greatest slope in the signal.
For less than1 LSB error the error needs to be less than the step size, which when $V_{pp}$ is 3.3V is 50$\mu$V.
However in the worst-case scenario when the input signal has a $V_{pp}$ of 3.3V and 100kHz the device can have a variation of 11.484mV which is far greater than the step size of 16bits.
In this case, the device can provide 8bit resolution since the equivalent step size for 8bit resolution would be 12.89mV and the error that could occur is smaller than that.

In the end, the thesis shows that the Teensy platform is capable for the task of recording cetacean vocalizations with more research into lowering the standard deviation for the sampling.
As human civilization grows and expands it impacts animal wildlife more and more, whether that be land- or water creatures.
Creating different ways as well as improving current methods to monitor those effects on the animals.
Becomes ever so more crucial, in an effort of ensuring the survival of some species and will contribute to a future where human impact on both the earth and wildlife is hopefully decreased to a point that is sustainable.


T

%%% Local Variables: 
%%% mode: latex
%%% TeX-master: "DEGREE-NAME-YEAR"
%%% End: 
